% theorem styles
\declaretheoremstyle[
    headfont=\sffamily\bfseries,
    bodyfont=\normalfont,
    notefont=\sffamily\mdseries,
    notebraces={(}{)},
    headformat={\NAME\space\NUMBER.\NOTE},
    headpunct={}
    ]{lightthm}

\declaretheorem[
    style=lightthm,
    name=Definition,
    refname={Definition,Definitions},
    within=section
]{definition}
\declaretheorem[
    style=lightthm,
    name=Definition,
    refname={Definition,Definitions},
    unnumbered
]{definition*}
\declaretheorem[
    style=lightthm,
    name=Example,
    refname={Example,Examples},
    sibling=definition
]{example}
\declaretheorem[
    style=lightthm,
    name=Exercise,
    refname={Exercise,Exercises},
    sibling=definition
]{exercise}
\declaretheorem[
    style=lightthm,
    name=Notation,
    refname={Notation,Notations},
    sibling=definition
]{notation}
\declaretheorem[
    style=lightthm,
    name=Caution,
    refname={Caution,Cautions},
    sibling=definition
]{caution}
\declaretheorem[
    style=lightthm,
    name=Remark,
    refname={Remark,Remarks},
    sibling=definition
]{remark}
\declaretheorem[
    style=lightthm,
    name=Redefinition,
    refname={Redefinition,Redefinitions},
    sibling=definition
]{redefinition}
% \declaretheorem[
%     style=lightthm,
%     name=전환점,
%     refname={전환점},
%     sibling=definition
% ]{turnpoint}
\declaretheorem[
    style=lightthm,
    name=Custom,
    refname={Custom,Customs},
    sibling=definition
]{custom}

\declaretheoremstyle[
    headfont=\sffamily\bfseries,
    bodyfont=\sffamily,
    notefont=\sffamily\bfseries,
    notebraces={(}{)},
    headformat={\NAME\space\NUMBER.\NOTE},
    headpunct={}
    ]{normalthm}

\declaretheorem[
    style=normalthm,
    name=Observation,
    refname={Observation,Observations},
    sibling=definition
]{observation}
\declaretheorem[
    style=normalthm,
    name=Observation,
    refname={Observation,Observations},
    unnumbered
]{observation*}
\declaretheorem[
    style=normalthm,
    name=Proposition,
    refname={Proposition,Propositions},
    sibling=definition
]{proposition}
\declaretheorem[
    style=normalthm,
    name=Lemma,
    refname={Lemma,Lemmas},
    sibling=definition
]{lemma}
\declaretheorem[
    style=normalthm,
    name=Theorem,
    refname={Theorem,Theorems},
    sibling=definition
]{theorem}
\declaretheorem[
    style=normalthm,
    name=Corollary,
    refname={Corollary,Corollaries},
    sibling=definition
]{corollary}
\declaretheorem[
    style=normalthm,
    name=Question,
    refname={Question,Questions},
    sibling=definition
]{question}

\declaretheoremstyle[
    headfont=\sffamily\bfseries,
    bodyfont=\normalfont,
    qed=\qedsymbol,
    headpunct={\::}
    ]{proof}

\declaretheorem[
    style=proof,
    name=Sketch of Proof,
    unnumbered
]{sketch}
\declaretheorem[
    style=proof,
    name=Proof,
    unnumbered
]{myproof}
\declaretheorem[
    style=proof,
    name=Solution,
    unnumbered
]{solution}
\newcommand\rightqed{{\hfill\qedhere}} % place qed at right side

\declaretheoremstyle[
    headfont=\sffamily\bfseries,
    notefont=\sffamily\bfseries,
    headformat={\NOTE},
    headpunct={\::},
    notebraces={\unskip}{},
    qed=\qedsymbol
    ]{miscellaneous}

\declaretheorem[
    style=miscellaneous,
    unnumbered
]{misc}