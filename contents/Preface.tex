\setchapter*{Preface}

This book is a collection
of personal (but not neccessarily private) notes
while studying in undergraduate level math.
This is intended not for strictness and rigor,
although I would like to achieve that kind of skill.
This is also not my creative content,
since (as any undergraduate textbooks would do so)
the curriculum of the subject is fairly concrete already.

The purpose of this book is somewhat vague;
it is my attempt to solve problems in some textbooks,
and also an (exaggerated) attempt to comprehend viewpoints of
the textbooks that I have learned from.
Also, the goal to update this note on a regular basis
is very motivating;
hence I have decided to write this note.

Exercises would mainly come from English and Korean books,
since those are the languages that I comprehend.
For citations and quotes here and there,
references from other languages may also be included.
See the (not yet made) bibliography for the full list of references.

\begin{itemize}
    \item In Part 1,
    we will cover the founding blocks of modern mathematics;
    logic and set theory.
    Not only is it important to know set theory `well',
    it would also be nice to cover
    what writing a proof means in mathematics,
    and how to write them.
    The subject will be mostly consisted of naive set theory,
    and briefly introduce ZFC set theory.
    The Axiom of Choice
    and construction of the natural numbers (up to rational numbers)
    will be introduced.

    After that,
    one must observe on how to deduce facts
    from axioms, definitions, previously proved facts and logic.
    I consider Hilbert's axioms on geometry
    as a good example to this,
    as well as an introduction for differential geometry
    in the (distant) future.

    \item In Part 2,
    the most basic concepts of analysis and algebra will appear;
    as well as some topology and differential equations.
    Materials in Part 1 are fundamental objects of modern mathematics,
    whereas materials in Part 2
    (real numbers, topology, linear algebra)
    offer
    tools granted from the beginning,
    how to internalize the meaning of them,
    and the language to observe some phenomena of basic objects.

    I think that introducing topology in an early stage is important
    in the sense of defining several notions of compactness,
    and why the open-cover definition is accepted at the end.

    Also, in the course of learning differential equations,
    the concept of Laplace transform is introduced;
    this may be a fun example of vector spaces,
    where functions themselves are viewed as vectors.

    \item In Part 3,
    some topics take a step deeper, while new topics arise.
    Linear algebra is extended as abstract algebra
    ---
    which, of course, is too broad for a single subject
    and is only the beginning in the undergraduate level
    ---
    and the concept of calculus in the real field
    is extended to multivariable calculus and complex analysis.

    \item In Part 4,
    things get more advanced,
    and we cover differential geometry,
    `real' (pun intended) analysis
    and numerical analysis.

    \item Unlike other parts,
    Part 5 is basically notes of what I've heard about
    more advanced parts but never properly learned myself.
    It may be a little messy.
\end{itemize}

Obviously,
the list I made above is very subject to change;
seeing it changing after all might be satisfying in some sense.