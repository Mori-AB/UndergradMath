\setsection{Proofs} \label{sec:proof}

In the preceeding sections,
we have ``proved'' propositions by some form of reasoning.
But what exactly is a proof in mathematics?
To understand its meaning in mathematics,
we should look into the defining properties of math.

Along other branches of science,
math is considered a `logical' study;
except for scarce moments of insights,
emotions can hardly interfere with mathematics
as mentioned at the beginning of \cref{sec:plogic}.

However, the defining properties of mathematics is that
(at the end of the day) it is a purely deductive branch of studies.
Many branches of studies mainly try to find governing laws of nature
by studying (a lot of) individual cases.
Newton `invented' the laws of motion and gravity
to explain various motions in real life,
and chemicists came up with the notion of atoms and molecules
to explain phenomenona in chemical reactions.
These are embraced by the academical society
because they best explain the surrounding environment,
not because they are logically fundamental in some sense.

However, a short period in studying math gives the impression that
propositions in mathematics are (largely) not from nature itself.
There is no `fundamental' nature that
surrounds us and becomes the ultimate goal of the whole study.
Instead, we merely agree on
some primitive sentences and a certain logical structure.
One would argue that
the logical structures came from our surroundings,
but mathematics have come a long way from our exact reality
since the invention of numbers.

This implies a crucial yet under-appreciated fact about mathematics:
Provided that we are on the same page
---
that is, agreeing on some primitive sentences,
some rules of inferences,
and a set of definitions
---
anyone can follow up any statements as long as they are not flawed.
(Of course, this does not necessarily mean that the path is easy.)
Thus mathematics is, in principle, all about writing good articles.
Finding out some properties of mathematical objects
and publishing them as articles or books
so that others can read about it
are the only activities in mathematics.

Now suppose that we found a phenomenon that seems true.
How do we convince the reader to believe this?
The answer is simple:
Since we and the reader share a common database of knowledge and logic,
if we derive our phenomenon from that database,
then the reader would agree that the phenomenon is true.
This is bascially what proof-writing means.

Note that since proofs are writings in principle,
there is no such thing as an objectively `good' proof
if it is logically true.
However, proofs that are hard to read \emph{do} exist;
for further content, see \cite{Knuth89}.
Some tips from the first sections that frequently arise are given below.
\begin{itemize}
    \item Separate symbols in different formula by words.
    \item Do not start sentences with a symbol.
    \item Do not use the symbols
    \(\iff\),
    \(\therefore\),
    \(\forall\),
    \(\exists\),
    \(\in\)
    in text;
    write their counterparts in words.
\end{itemize}