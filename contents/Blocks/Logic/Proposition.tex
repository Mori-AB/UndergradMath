\setsection{Propositional Logic} \label{sec:plogic}

Propositional logic is one of the simplest forms of logic.
It deals with sentences such as
`If it rains, then we stay at home.
It rains.
Thus, we will stay at home.'
More formally speaking,
propositional logic is a branch of logic that deals with
propositions
and relations between them.

When we say propositions,
we mean a sentence that is either true or false;
we do not consider sentences that are
`somewhat true,'
`both true and false,'
or `neither true nor false.'
In the usual sense,
this eliminates sentences that require subjective decisions.

Now how do we define what propositions are?

\begin{definition}[Propositional Formula]
    \label{def:propformula}
    Propositional logic are described by three components:
    atomic propositions,
    connectives,
    and parentheses.

    A \define{propositional formula} is either
    an atomic proposition,
    or propositions (not necessarily atomic) connected by connectives.
    As opposed to atomic propositions,
    the latter are also called compound propositions.
\end{definition}

\begin{remark}
    Typically, atomic propositions are denoted as \(p,q,r,\ldots\).
    These are sometimes called propositional symbols.
    Combined with the notations for connectives,
    it is possible to view propositions as \emph{formal strings},
    hence not necessarily requiring set theory
    as a prerequisite for logic.
\end{remark}

There are five types of connectives:
negation,
conjunction,
disjunction,
conditional,
and biconditional.
To see the effect of each connectives,
it is useful to see their \textdef{truth tables}.
\begin{itemize}
    \item The \define{negation} of a proposition \(A\),
    denoted by \(\denote{\neg} A\),
    is a proposition which always have the other truth value
    besides one of \(A\).
    Note that even though negation requires a single proposition,
    it is still considered a ``connect''ive.
    \[
        \begin{array}{c}\toprule
            A \\ \midrule
            \top \\
            \bot \\ \bottomrule
        \end{array}
        \;
        \begin{array}{c}\toprule
            \neg A \\ \midrule
            \bot \\
            \top \\ \bottomrule
        \end{array}
    \]

    \item The \define{conjunction} of propositions \(A\) and \(B\),
    denoted by \(A\denote{\wedge} B\),
    is a proposition which is true
    only when \(A\) and \(B\) are both true,
    and false otherwise.
    % It is possible to extend this definition for \(n\) propositions,
    % where \(p_1\wedge p_2\wedge\dots\wedge p_n\) is true
    % only when \(p_1,\dots,p_n\) are all true,
    % and false otherwise.
    \[
        \begin{array}{c c}\toprule
            A & B \\ \midrule
            \top & \top \\
            \top & \bot \\
            \bot & \top \\
            \bot & \bot \\ \bottomrule
        \end{array}
        \;
        \begin{array}{c}\toprule
            A\wedge B \\ \midrule
            \top \\
            \bot \\
            \bot \\
            \bot \\ \bottomrule
        \end{array}
    \]

    \item The \define{disjunction} of propositions \(A\) and \(B\),
    denoted by \(A\denote{\vee} B\),
    is a proposition which is false
    only when \(A\) and \(B\) are both false,
    and true otherwise.
    % It is possible to extend this definition for \(n\) propositions,
    % where \(p_1\wedge p_2\wedge\dots\wedge p_n\) is true
    % only when \(p_1,\dots,p_n\) are all true,
    % and false otherwise.
    \[
        \begin{array}{c c}\toprule
            A & B \\ \midrule
            \top & \top \\
            \top & \bot \\
            \bot & \top \\
            \bot & \bot \\ \bottomrule
        \end{array}
        \;
        \begin{array}{c}\toprule
            A\vee B \\ \midrule
            \top \\
            \top \\
            \top \\
            \bot \\ \bottomrule
        \end{array}
    \]
    It should be noted that
    the word ``or'' in mathematics is nonexclusive;
    that is, \phrase{\(A\vee B\) and \(A\)}
    does not necessarily imply \(\neg B\).
    This is sometimes not the case in everyday lives.
    In a crossroad, one would not expect to take both routes
    ---
    only one route is allowed.

    \item The \define{conditional} of propositions \(A\) and \(B\)
    make up the phrase \phrase{if \(A\), then \(B\).}
    Propositions \(A\) and \(B\) are each called
    the \define{antecedent} and \define{consequent}.
    Denoted by \(A\denote{\to}B\),
    it is the logical equivalent of the proposition
    \(\neg A\vee B\).
    As for why one would define the conditional like this,
    consider a daily-life sentence ``if it rains, then we stay home.''
    If it does rain,
    then the sentence is fullfilled only when one stays at home.
    However, if it does not rain,
    then it does not really matter whether one stays at home.
    \[
        \begin{array}{c c}\toprule
            A & B \\ \midrule
            \top & \top \\
            \top & \bot \\
            \bot & \top \\
            \bot & \bot \\ \bottomrule
        \end{array}
        \;
        \begin{array}{c}\toprule
            A\to B \\ \midrule
            \top \\
            \bot \\
            \top \\
            \top \\ \bottomrule
        \end{array}
    \]
    One other way to think about this is
    to think when the above statement is false.
    If it rains, then it is clearly false if one goes outside.
    However, if it does not rain,
    then whatever you do technically there is nothing wrong.
    Life becomes easier if we consider these cases true.

    \item The \define{biconditional} of propositions \(A\) and \(B\)
    make up the phrase ``\(A\) if and only if \(B\),''
    or ``\(A\) iff \(B\)'' in short.
    Denoted as \(A\denote{\iff}B\),
    it is only true when \(A\) and \(B\) have the same truth value.
    \[
        \begin{array}{c c} \toprule
            A & B \\ \midrule
            \top & \top \\
            \top & \bot \\
            \bot & \top \\
            \bot & \bot \\ \bottomrule            
        \end{array}
        \;
        \begin{array}{c} \toprule
            A\iff B \\ \midrule
            \top \\
            \bot \\
            \bot \\
            \top \\ \bottomrule            
        \end{array}
    \]
\end{itemize}

In reality (mostly in proof-writing),
some more definitions are required.

\begin{definition}[Conditionals]
    \label{def:conditional}
    The propositional formula \(A\to B\)
    is sometimes called an \define{implication} from \(A\) to \(B\).
    Also, given an implication \(A\to B\),
    the implication \(B\to A\) is called
    the \define{converse} of \(A\to B\),
    and the implication \(\neg B\to\neg A\) is called
    the \define{contrapositive} of \(A\to B\).
\end{definition}

In the process of representing the effect of connectives,
we made use of truth tables.
Though not a common convention,
here we make a distinction between
independent and dependent propositions
by separating the tables for them.
The table in the left represents independent propositions,
and the table in the right represents dependent ones.
But what exactly does each row in the table mean?

In each row of a truth table,
a possibility of truth values associated to independent propositions
are listed.
One can view this as assigning truth values
to each independent propositions.

\begin{definition}[Interpretation]
    \label{def:interpret}    
    Generally speaking,
    a \define{propositional variable}
    is a variable that has
    truth values \(\top\) and \(\bot\) as inputs.
    Now given a collection \(\mathcal P\) of proposition variables,
    an \define{interpretation} of \(\mathcal P\)
    is a assignment of truth values for each propositional symbol.
\end{definition}

\begin{example}
    \begin{enumerate}[label=(\alph*)]
        \item In the truth table for negation,
        there is only one propositional variable; namely, \(A\).
        Naturally there are only two interpretations,
        hence only two rows in the table.

        \item In the truth table for conjunction,
        there are two propositional variables; namely, \(A\) and \(B\).
        Naturally there are only four interpretations,
        hence four rows in the table.

        \item Given a finite collection of proposition variables
        \(\{A_1,\dots,A_n\}\),
        there are \(2^n\) possible interpretations.
    \end{enumerate}
\end{example}

Now we have named the left half of the given truth tables;
it is time for the right half of them.

\begin{definition}[Truth of a Proposition]
    \label{def:truth}
    Let \(A\) be a proposition
    determined by some collection \(\mathcal P\)
    of propositional variables,
    and \(I\) be an interpretation of \(\mathcal P\).
    Then \(A\) is said to be \define{true} under \(I\)
    if \(A\) is assigned the truth value \(\top\)
    with interpretation \(I\);
    if \(A\) is assigned the truth value \(\bot\),
    then \(A\) is said to be \define{false} under \(I\).

    A proposition that is true under all interpretations
    is called a \define{tautology}.
\end{definition}

As for the order of assigning values, the familiar rules apply;
those inside parentheses are assigned first,
then those outside parentheses.
Also, to avoid using too many parentheses,
we make the conventions that
\(\neg\) comes faster than \(\wedge\) or \(\vee\),
and those two come faster than \(\to\) and \(\iff\).
Note that
there are no orders between \(\wedge\) and \(\vee\),
and between \(\to\) and \(\iff\).

Given the definition of interpretations,
a seemingly natural question arises:
Does there exist propositions that always have the same truth value
under any interpretations?
The answer is undoubtedly yes;
we have seen it in the definition of conditionals.
More explicitly,
comparing the truth tables of \(p\to q\) and \(\neg p\vee q\)
gives the following:
\[
    \begin{array}{c c}\toprule
        p & q \\ \midrule
        \top & \top \\
        \top & \bot \\
        \bot & \top \\
        \bot & \bot \\ \bottomrule
    \end{array}
    \;
    \begin{array}{c c}\toprule
        \neg p & \neg p\vee q \\ \midrule
        \bot & \top \\
        \bot & \bot \\
        \top & \top \\
        \top & \top \\ \bottomrule
    \end{array}
    \;
    \begin{array}{c} \toprule
        p\to q \\ \midrule
        \top \\
        \bot \\
        \top \\
        \top \\ \bottomrule
    \end{array}
\]

Note that
the two formulas \(\neg p\vee q\) and \(p\to q\)
are technically not the same;
one has a negation symbol, the other has not.
However, they actually mean the same at every case.
This somewhat subtle difference is described below.

\begin{definition}[Syntax vs Semantics]
    \label{def:synsem}
    Two propositional formulas \(A\) and \(B\)
    are called the same in the \define{syntactic} sense
    if \(A\) and \(B\) are the same string,
    and is denoted by \(A\equiv B\).

    On the other hand,
    two propositional formulas \(A\) and \(B\)
    are called the same in the \define{semantic} sense
    if \(A\) and \(B\) are assigned the same truth value
    for every possible interpretation \(I\),
    and is denoted by \(A\Iff B\).
\end{definition}

The difference between \(\iff\) and \(\Iff\)
is that \(\iff\) is a connective,
while \(\Iff\) is a sentence.
Of course, the distinction is not crucial,
since if \(A\Iff B\) for propositions \(A\) and \(B\),
then \(A\iff B\) is a tautology.

Further examples of tautologies are as follows.

\begin{example}
    \label{exm:taut}
    Given propositions \(A,B,C\), the following are tautologies.
    \begin{nlist}
        \item \(A\wedge A\iff A\),
        \(A\vee A\iff A\)
        \hfill (Idempotency)

        \item \(A\wedge B\iff B\wedge A\),
        \(A\vee B\iff B\vee A\),
        \((A\iff B)\iff(B\iff A)\)
        \hfill (Commutativity)

        \item \((A\wedge B)\wedge C\iff A\wedge (B\wedge C)\),
        \((A\vee B)\vee C\iff A\vee(B\vee C)\)
        \hfill (Associativity)

        \item \(A\wedge B\to A\),
        \(A\to A\vee B\)

        \item \((A\wedge B)\vee A\iff A\),
        \((A\vee B)\wedge A\iff A\)
        \hfill (Absorption)

        \item \(\neg\neg A\iff A\)
        \hfill (Double negation)

        \item \(\neg(A\wedge B)\iff\neg A\vee\neg B\),
        \(\neg(A\vee B)\iff\neg A\wedge\neg B\)
        \hfill (De Morgan laws)

        \item \(A\wedge(B\vee C)\iff(A\wedge B)\vee(A\wedge C)\),
        \(A\vee(B\wedge C)\iff(A\vee B)\wedge(A\vee C)\)
        \hfill (Distributivity)
        
        \item \((A\to B)\iff(\neg B\to\neg A)\)
        \hfill (Contrapositive)

        \item \((A\to B)\wedge(B\to C)\to(A\to C)\)
        \hfill (Transitivity)

        \item \(\neg (A\to B)\iff A\wedge\neg B\)

        \item \(((A\to B)\to A)\to A\)
        \hfill (Peirce)
    \end{nlist}
\end{example}

Note that for calculating truth values of propositions,
there exists a certain `hierarchy' made by parentheses.
In the above, the hierarchy is treated mentally,
but for more complex propositions, a modification is useful.

Consider the following truth table:
\[
    \begin{array}{c c}
        \toprule
        A & B \\
        \midrule
        \top & \top \\
        \top & \bot \\
        \bot & \top \\
        \bot & \bot \\
        \midrule
        \multicolumn{2}{c}{\text{Hierarchy}} \\
        \bottomrule
    \end{array}
    \;
    \begin{array}{c c c c c c c} \toprule
        ( & A & \wedge & B & ) & \wedge & A \\ \midrule
         & \top & \top & \top & & \top & \top \\
         & \top & \bot & \bot & & \bot & \top \\
         & \bot & \bot & \top & & \bot & \bot \\
         & \bot & \bot & \bot & & \bot & \bot \\ \midrule
         & 1 & 2 & 1 &  & \mathbf 3 & 1 \\ \bottomrule
    \end{array}
\]
This table is written in the following steps:
\begin{itemize}
    \item In the left table,
    write down all possible interpretations.
    
    \item In the bottom row of the right table,
    write down the hierachy steps of each proposition.
    If the proposition of the cell is atomic, then it becomes 1.
    If the proposition is compound,
    find the main connective
    and the step is 1 larger than the maximum step among its parts.
    It is visually helpful
    to determine the truth value of the entire formula
    by writing the largest number in bold.
    
    \item If the hierarchy step is 1,
    then we simply copy the values from the left table.
    
    \item Fill in the truth values of compound connectives
    starting from lower values of steps.
    The required values should be given from the steps before.
\end{itemize}

With the improved method, we now verify \cref{exm:taut}.
\begin{myproof}
    \begin{nlist}
        \item For a proposition \(A\),
        the truth tables are computed as below.
        \[
            \begin{array}{c}
                \toprule
                A \\
                \midrule
                \top \\
                \bot \\
                \midrule
                \multicolumn{1}{c}{\text{Hierarchy}} \\
                \bottomrule
            \end{array}
            \;
            \begin{array}{c c c c c}
                \toprule
                A & \wedge & A & \iff & A \\
                \midrule
                \top & \top & \top & \top & \top \\
                \bot & \bot & \bot & \top & \bot \\ 
                \midrule
                1 & 2 & 1 & \mathbf 3 & 1 \\
                \bottomrule
            \end{array}
            \;
            \begin{array}{c c c c c}
                \toprule
                A & \vee & A & \iff & A \\
                \midrule
                \top & \top & \top & \top & \top \\
                \bot & \bot & \bot & \top & \bot \\ 
                \midrule
                1 & 2 & 1 & \mathbf 3 & 1 \\
                \bottomrule
            \end{array}
        \]

        \item For propositions \(A\) and \(B\),
        the truth tables are computed as below.
        \[
            \begin{array}{c c}
                \toprule
                A & B \\
                \midrule
                \top & \top \\
                \top & \bot \\
                \bot & \top \\
                \bot & \bot \\
                \midrule
                \multicolumn{2}{c}{\text{Hierarchy}} \\
                \bottomrule
            \end{array}
            \;
            \begin{array}{c c c c c c c}
                \toprule
                A & \wedge & B & \iff & B & \wedge & A \\
                \midrule
                \top & \top & \top & \top & \top & \top & \top \\ 
                \top & \bot & \bot & \top & \bot & \bot & \top \\ 
                \bot & \bot & \top & \top & \top & \bot & \bot \\ 
                \bot & \bot & \bot & \top & \bot & \bot & \bot \\ 
                \midrule
                1 & 2 & 1 & \mathbf 3 & 1 & 2 & 1 \\
                \bottomrule
            \end{array}
            \;
            \begin{array}{c c c c c c c}
                \toprule
                A & \vee & B & \iff & B & \vee & A \\
                \midrule
                \top & \top & \top & \top & \top & \top & \top \\ 
                \top & \top & \bot & \top & \bot & \top & \top \\ 
                \bot & \top & \top & \top & \top & \top & \bot \\ 
                \bot & \bot & \bot & \top & \bot & \bot & \bot \\ 
                \midrule
                1 & 2 & 1 & \mathbf 3 & 1 & 2 & 1 \\
                \bottomrule
            \end{array}
        \]
        \[
            \begin{array}{c c}
                \toprule
                A & B \\
                \midrule
                \top & \top \\
                \top & \bot \\
                \bot & \top \\
                \bot & \bot \\
                \midrule
                \multicolumn{2}{c}{\text{Hierarchy}} \\
                \bottomrule
            \end{array}
            \;
            \begin{array}{c c c c c c c c c c c}
                \toprule
                ( & A & \iff & B & )
                & \iff & ( & A & \iff & B & ) \\
                \midrule
                & \top & \top & \top & 
                & \top &  & \top & \top & \top &  \\
                & \top & \bot & \bot & 
                & \top &  & \top & \bot & \bot &  \\
                & \bot & \bot & \top & 
                & \top &  & \bot & \bot & \top &  \\
                & \bot & \top & \bot & 
                & \top &  & \bot & \top & \bot &  \\
                \midrule
                & 1 & 2 & 1 & 
                & \mathbf 3 &  & 1 & 2 & 1  & \\
                \bottomrule
            \end{array}
        \]

        \item For propositions \(A\), \(B\) and \(C\),
        the truth tables are computed as below.
        \begin{gather*}
            \begin{array}{c c c}
                \toprule
                A & B & C \\
                \midrule
                \top & \top & \top \\
                \top & \top & \bot \\
                \top & \bot & \top \\
                \top & \bot & \bot \\
                \bot & \top & \top \\
                \bot & \top & \bot \\
                \bot & \bot & \top \\
                \bot & \bot & \bot \\
                \midrule
                \multicolumn{3}{c}{\text{Hierarchy}} \\
                \bottomrule
            \end{array}
            \;
            \begin{array}{c c c c c c c c c c c c c c c}
                \toprule
                ( & A & \wedge & B & ) & \wedge & C
                & \iff
                & A & \wedge & ( & B & \wedge & C & ) \\
                \midrule
                 & \top & \top & \top & & \top & \top
                & \top
                & \top & \top & & \top & \top & \top & \\
                & \top & \top & \top & & \bot & \bot
                & \top
                & \top & \bot & & \top & \bot & \bot & \\
                & \top & \bot & \bot & & \bot & \top
                & \top
                & \top & \bot & & \bot & \bot & \top & \\
                & \top & \bot & \bot & & \bot & \bot
                & \top
                & \top & \bot & & \bot & \bot & \bot & \\
                & \bot & \bot & \top & & \bot & \top
                & \top
                & \bot & \bot & & \top & \top & \top & \\
                & \bot & \bot & \top & & \bot & \bot
                & \top
                & \bot & \bot & & \top & \bot & \bot & \\
                & \bot & \bot & \bot & & \bot & \top
                & \top
                & \bot & \bot & & \bot & \bot & \top & \\
                & \bot & \bot & \bot & & \bot & \bot
                & \top
                & \bot & \bot & & \bot & \bot & \bot & \\
                \midrule
                 & 1 & 2 & 1 &  & 3 & 1
                & \mathbf 4
                & 1 & 3 & & 1 & 2 & 1 \\ \bottomrule
            \end{array} \\
            \begin{array}{c c c}
                \toprule
                A & B & C \\
                \midrule
                \top & \top & \top \\
                \top & \top & \bot \\
                \top & \bot & \top \\
                \top & \bot & \bot \\
                \bot & \top & \top \\
                \bot & \top & \bot \\
                \bot & \bot & \top \\
                \bot & \bot & \bot \\
                \midrule
                \multicolumn{3}{c}{\text{Hierarchy}} \\
                \bottomrule
            \end{array}
            \;
            \begin{array}{c c c c c c c c c c c c c c c}
                \toprule
                ( & A & \vee & B & ) & \vee & C
                & \iff
                & A & \vee & ( & B & \vee & C & ) \\
                \midrule
                 & \top & \top & \top & & \top & \top
                & \top
                & \top & \top & & \top & \top & \top & \\
                & \top & \top & \top & & \top & \bot
                & \top
                & \top & \top & & \top & \top & \bot & \\
                & \top & \top & \bot & & \top & \top
                & \top
                & \top & \top & & \bot & \top & \top & \\
                & \top & \top & \bot & & \top & \bot
                & \top
                & \top & \top & & \bot & \bot & \bot & \\
                & \bot & \top & \top & & \top & \top
                & \top
                & \bot & \top & & \top & \top & \top & \\
                & \bot & \top & \top & & \top & \bot
                & \top
                & \bot & \top & & \top & \top & \bot & \\
                & \bot & \bot & \bot & & \top & \top
                & \top
                & \bot & \top & & \bot & \top & \top & \\
                & \bot & \bot & \bot & & \bot & \bot
                & \top
                & \bot & \bot & & \bot & \bot & \bot & \\
                \midrule
                 & 1 & 2 & 1 &  & 3 & 1
                & \mathbf 4
                & 1 & 3 & & 1 & 2 & 1 \\ \bottomrule
            \end{array}
        \end{gather*}

        \item For propositions \(A\) and \(B\),
        the truth tables are computed as below.
        \[
            \begin{array}{c c}
                \toprule
                A & B \\
                \midrule
                \top & \top \\
                \top & \bot \\ 
                \bot & \top \\
                \bot & \bot \\
                \midrule
                \multicolumn{2}{c}{\text{Hierarchy}} \\
                \bottomrule
            \end{array}
            \;
            \begin{array}{c c c c c}
                \toprule
                A & \wedge & B & \to & A \\
                \midrule
                \top & \top & \top & \top & \top \\
                \top & \bot & \bot & \top & \top \\
                \bot & \bot & \top & \top & \bot \\
                \bot & \bot & \bot & \top & \bot \\
                \midrule
                1 & 2 & 1 & \mathbf 3 & 1 \\
                \bottomrule
            \end{array}
            \;
            \begin{array}{c c c c c}
                \toprule
                A & \to & A & \vee & B \\
                \midrule
                \top & \top & \top & \top & \top \\
                \top & \top & \top & \top & \bot \\
                \bot & \top & \bot & \top & \top \\
                \bot & \top & \bot & \bot & \bot \\
                \midrule
                1 & \mathbf 3 & 1 & 2 & 1 \\
                \bottomrule
            \end{array}
        \]

        \item For propositions \(A\) and \(B\),
        the truth tables are computed as below.
        \[
            \begin{array}{c c}
                \toprule
                A & B \\
                \midrule
                \top & \top \\
                \top & \bot \\
                \bot & \top \\
                \bot & \bot \\
                \midrule
                \multicolumn{2}{c}{\text{Hierarchy}} \\
                \bottomrule
            \end{array}
            \;
            \begin{array}{c c c c c c c c c}
                \toprule
                ( & A & \wedge & B & ) & \vee & A & \iff & A \\
                \midrule
                 & \top & \top & \top &  & \top & \top & \top & \top \\
                 & \top & \bot & \bot &  & \top & \top & \top & \top \\
                 & \bot & \bot & \top &  & \bot & \bot & \top & \bot \\
                 & \bot & \bot & \bot &  & \bot & \bot & \top & \bot \\
                \midrule
                 & 1 & 2 & 1 &  & 3 & 1 & \mathbf 4 & 1 \\
                \bottomrule
            \end{array}
            \;
            \begin{array}{c c c c c c c c c}
                \toprule
                ( & A & \vee & B & ) & \wedge & A & \iff & A \\
                \midrule
                 & \top & \top & \top &  & \top & \top & \top & \top \\
                 & \top & \top & \bot &  & \top & \top & \top & \top \\
                 & \bot & \top & \top &  & \bot & \bot & \top & \bot \\
                 & \bot & \bot & \bot &  & \bot & \bot & \top & \bot \\
                \midrule
                 & 1 & 2 & 1 &  & 3 & 1 & \mathbf 4 & 1 \\
                \bottomrule
            \end{array}
        \]
    \end{nlist}

    Although worthwhile,
    writing down all the truth tables in \LaTeX{} is quite tedious;
    one could also make a program
    to automatically generate truth tables.
    The results here are mainly generated by Python.

    \begin{nlist}[resume]
        \item For a proposition \(A\),
        the truth table is computed as below.
        \[
            \begin{array}{c}
                \toprule
                A \\
                \midrule
                \top \\
                \bot \\
                \midrule
                \multicolumn{1}{c}{\text{Hierarchy}} \\
                \bottomrule
            \end{array}
            \;
            \begin{array}{c c c c c}
                \toprule
                \neg & \neg & A & \iff & A \\
                \midrule
                \top & \bot & \top & \top & \top \\
                \bot & \top & \bot & \top & \bot \\
                \midrule
                3 & 2 & 1 & \mathbf 4 & 1 \\
                \bottomrule
            \end{array}
        \]

        \item For propositions \(A\) and \(B\),
        the truth tables are computed as below.
        \begin{gather*}
            \begin{array}{c c}
                \toprule
                A & B \\
                \midrule
                \top & \top \\
                \top & \bot \\
                \bot & \top \\
                \bot & \bot \\
                \midrule
                \multicolumn{2}{c}{\text{Hierarchy}} \\
                \bottomrule
            \end{array}
            \;
            \begin{array}{c c c c c c c c c c c c}
                \toprule
                \neg & ( & A & \wedge & B & )
                & \iff
                & \neg & A & \vee & \neg & B \\
                \midrule
                \bot & & \top & \top & \top & 
                & \top
                & \bot & \top & \bot & \bot & \top \\
                \top & & \top & \bot & \bot & 
                & \top
                & \bot & \top & \top & \top & \bot \\
                \top & & \bot & \bot & \top & 
                & \top
                & \top & \bot & \top & \bot & \top \\
                \top & & \bot & \bot & \bot & 
                & \top
                & \top & \bot & \top & \top & \bot \\
                \midrule
                3 &  & 1 & 2 & 1 &  & \mathbf 4 & 2 & 1 & 3 & 2 & 1 \\
                \bottomrule
            \end{array} \\
            \begin{array}{c c}
                \toprule
                A & B \\
                \midrule
                \top & \top \\
                \top & \bot \\
                \bot & \top \\
                \bot & \bot \\
                \midrule
                \multicolumn{2}{c}{\text{Hierarchy}} \\
                \bottomrule
            \end{array}
            \;
            \begin{array}{c c c c c c c c c c c c}
                \toprule
                \neg & ( & A & \vee & B & ) & \iff & \neg & A & \wedge & \neg & B \\
                \midrule
                \bot &  & \top & \top & \top &  & \top & \bot & \top & \bot & \bot & \top \\
                \bot &  & \top & \top & \bot &  & \top & \bot & \top & \bot & \top & \bot \\
                \bot &  & \bot & \top & \top &  & \top & \top & \bot & \bot & \bot & \top \\
                \top &  & \bot & \bot & \bot &  & \top & \top & \bot & \top & \top & \bot \\
                \midrule
                3 &  & 1 & 2 & 1 &  & \mathbf 4 & 2 & 1 & 3 & 2 & 1 \\
                \bottomrule
            \end{array}
        \end{gather*}

        \item For propositions \(A\), \(B\) and \(C\),
        the truth tables are computed as below.
        \begin{gather*}
            \begin{array}{c c c}
                \toprule
                A & B & C \\
                \midrule
                \top & \top & \top \\
                \top & \top & \bot \\
                \top & \bot & \top \\
                \top & \bot & \bot \\
                \bot & \top & \top \\
                \bot & \top & \bot \\
                \bot & \bot & \top \\
                \bot & \bot & \bot \\
                \midrule
                \multicolumn{3}{c}{\text{Hierarchy}} \\
                \bottomrule
            \end{array}
            \;
            \begin{array}{c c c c c c c c c c c c c c c c c c c}
                \toprule
                A & \wedge & ( & B & \vee & C & )
                & \iff
                & ( & A & \wedge & B & )
                & \vee & ( & A & \wedge & C & ) \\
                \midrule
                \top & \top &  & \top & \top & \top & 
                & \top
                &  & \top & \top & \top & 
                & \top &  & \top & \top & \top &  \\
                \top & \top &  & \top & \top & \bot & 
                & \top
                &  & \top & \top & \top & 
                & \top &  & \top & \bot & \bot &  \\
                \top & \top &  & \bot & \top & \top & 
                & \top
                &  & \top & \bot & \bot & 
                & \top &  & \top & \top & \top &  \\
                \top & \bot &  & \bot & \bot & \bot & 
                & \top
                &  & \top & \bot & \bot & 
                & \bot &  & \top & \bot & \bot &  \\
                \bot & \bot &  & \top & \top & \top & 
                & \top
                &  & \bot & \bot & \top & 
                & \bot &  & \bot & \bot & \top &  \\
                \bot & \bot &  & \top & \top & \bot & 
                & \top
                &  & \bot & \bot & \top &
                & \bot &  & \bot & \bot & \bot &  \\
                \bot & \bot &  & \bot & \top & \top & 
                & \top
                &  & \bot & \bot & \bot & 
                & \bot &  & \bot & \bot & \top &  \\
                \bot & \bot &  & \bot & \bot & \bot & 
                & \top
                &  & \bot & \bot & \bot & 
                & \bot &  & \bot & \bot & \bot &  \\
                \midrule
                1 & 3 &  & 1 & 2 & 1 & 
                & \mathbf 4 & 
                & 1 & 2 & 1 & 
                & 3 &  & 1 & 2 & 1  & \\
                \bottomrule
            \end{array} \\
            \begin{array}{c c c}
                \toprule
                A & B & C \\
                \midrule
                \top & \top & \top \\
                \top & \top & \bot \\
                \top & \bot & \top \\
                \top & \bot & \bot \\
                \bot & \top & \top \\
                \bot & \top & \bot \\
                \bot & \bot & \top \\
                \bot & \bot & \bot \\
                \midrule
                \multicolumn{3}{c}{\text{Hierarchy}} \\
                \bottomrule
            \end{array}
            \;
            \begin{array}{c c c c c c c c c c c c c c c c c c c}
                \toprule
                A & \vee & ( & B & \wedge & C & )
                & \iff
                & ( & A & \vee & B & )
                & \wedge & ( & A & \vee & C & ) \\
                \midrule
                \top & \top &  & \top & \top & \top & 
                & \top
                &  & \top & \top & \top & 
                & \top &  & \top & \top & \top &  \\
                \top & \top &  & \top & \bot & \bot & 
                & \top
                &  & \top & \top & \top & 
                & \top &  & \top & \top & \bot &  \\
                \top & \top &  & \bot & \bot & \top & 
                & \top
                &  & \top & \top & \bot & 
                & \top &  & \top & \top & \top &  \\
                \top & \top &  & \bot & \bot & \bot & 
                & \top
                &  & \top & \top & \bot &
                & \top &  & \top & \top & \bot &  \\
                \bot & \top &  & \top & \top & \top & 
                & \top
                &  & \bot & \top & \top & 
                & \top &  & \bot & \top & \top &  \\
                \bot & \bot &  & \top & \bot & \bot & 
                & \top
                &  & \bot & \top & \top & 
                & \bot &  & \bot & \bot & \bot &  \\
                \bot & \bot &  & \bot & \bot & \top & 
                & \top
                &  & \bot & \bot & \bot & 
                & \bot &  & \bot & \top & \top &  \\
                \bot & \bot &  & \bot & \bot & \bot & 
                & \top
                &  & \bot & \bot & \bot & 
                & \bot &  & \bot & \bot & \bot &  \\
                \midrule
                1 & 3 &  & 1 & 2 & 1 & 
                & \mathbf 4
                &  & 1 & 2 & 1 & 
                & 3 &  & 1 & 2 & 1  & \\
                \bottomrule
            \end{array}
        \end{gather*}

        \item For propositions \(A\) and \(B\),
        the truth table is computed as below.
        \[
            \begin{array}{c c}
                \toprule
                A & B \\
                \midrule
                \top & \top \\
                \top & \bot \\
                \bot & \top \\
                \bot & \bot \\
                \midrule
                \multicolumn{2}{c}{\text{Hierarchy}} \\
                \bottomrule
            \end{array}
            \;
            \begin{array}{c c c c c c c c c c c c c}
                \toprule
                ( & A & \to & B & ) &
                \iff & ( & \neg & B & \to & \neg & A & ) \\
                \midrule
                 & \top & \top & \top & 
                 & \top &  & \bot & \top & \top & \bot & \top &  \\
                 & \top & \bot & \bot & 
                 & \top &  & \top & \bot & \bot & \bot & \top &  \\
                 & \bot & \top & \top & 
                 & \top &  & \bot & \top & \top & \top & \bot &  \\
                 & \bot & \top & \bot & 
                 & \top &  & \top & \bot & \top & \top & \bot &  \\
                \midrule
                 & 1 & 2 & 1 & 
                & \mathbf 4 &  & 2 & 1 & 3 & 2 & 1  & \\
                \bottomrule
            \end{array}
        \]

        \item For propositions \(A\), \(B\) and \(C\),
        the truth table is computed as below.
        \[
            \begin{array}{c c c}
                \toprule
                A & B & C \\
                \midrule
                \top & \top & \top \\
                \top & \top & \bot \\
                \top & \bot & \top \\
                \top & \bot & \bot \\
                \bot & \top & \top \\
                \bot & \top & \bot \\
                \bot & \bot & \top \\
                \bot & \bot & \bot \\
                \midrule
                \multicolumn{3}{c}{\text{Hierarchy}} \\
                \bottomrule
            \end{array}
            \;
            \begin{array}{c c c c c c c c c c c c c c c c c}
                \toprule
                ( & A & \to & B & )
                & \wedge & ( & B & \to & C & )
                & \to & ( & A & \to & C & ) \\
                \midrule
                 & \top & \top & \top & 
                 & \top &  & \top & \top & \top & 
                 & \top &  & \top & \top & \top &  \\
                 & \top & \top & \top & 
                 & \bot &  & \top & \bot & \bot & 
                 & \top &  & \top & \bot & \bot &  \\
                 & \top & \bot & \bot & 
                 & \bot &  & \bot & \top & \top & 
                 & \top &  & \top & \top & \top &  \\
                 & \top & \bot & \bot & 
                 & \bot &  & \bot & \top & \bot & 
                 & \top &  & \top & \bot & \bot &  \\
                 & \bot & \top & \top & 
                 & \top &  & \top & \top & \top & 
                 & \top &  & \bot & \top & \top &  \\
                 & \bot & \top & \top & 
                 & \bot &  & \top & \bot & \bot & 
                 & \top &  & \bot & \top & \bot &  \\
                 & \bot & \top & \bot & 
                 & \top &  & \bot & \top & \top & 
                 & \top &  & \bot & \top & \top &  \\
                 & \bot & \top & \bot & 
                 & \top &  & \bot & \top & \bot & 
                 & \top &  & \bot & \top & \bot &  \\
                \midrule
                 & 1 & 2 & 1 & 
                & 3 &  & 1 & 2 & 1 &
                & \mathbf 4 &  & 1 & 2 & 1  & \\
                \bottomrule
            \end{array}
        \]

        \item For propositions \(A\) and \(B\),
        the truth table is computed as below.
        \[
            \begin{array}{c c}
                \toprule
                A & B \\
                \midrule
                \top & \top \\
                \top & \bot \\
                \bot & \top \\
                \bot & \bot \\
                \midrule
                \multicolumn{2}{c}{\text{Hierarchy}} \\
                \bottomrule
            \end{array}
            \;
            \begin{array}{c c c c c c c c c c c}
                \toprule
                \neg & ( & A & \to & B & ) &
                \iff & A & \wedge & \neg & B \\
                \midrule
                \bot &  & \top & \top & \top & 
                & \top & \top & \bot & \bot & \top \\
                \top &  & \top & \bot & \bot & 
                & \top & \top & \top & \top & \bot \\
                \bot &  & \bot & \top & \top & 
                & \top & \bot & \bot & \bot & \top \\
                \bot &  & \bot & \top & \bot & 
                & \top & \bot & \bot & \top & \bot \\
                \midrule
                3 &  & 1 & 2 & 1 & 
                & \mathbf 4 & 1 & 3 & 2 & 1 \\
                \bottomrule
            \end{array}
        \]

        \item For propositions \(A\) and \(B\),
        the truth table is computed as below.
        \[
            \begin{array}{c c}
                \toprule
                A & B \\
                \midrule
                \top & \top \\
                \top & \bot \\
                \bot & \top \\
                \bot & \bot \\
                \midrule
                \multicolumn{2}{c}{\text{Hierarchy}} \\
                \bottomrule
            \end{array}
            \;
            \begin{array}{c c c c c c c c c c c}
                \toprule
                ( & ( & A & \to & B & ) & \to & A & )
                & \to & A \\
                \midrule
                 &  & \top & \top & \top &  & \top & \top & 
                 & \top & \top \\
                 &  & \top & \bot & \bot &  & \top & \top & 
                 & \top & \top \\
                 &  & \bot & \top & \top &  & \bot & \bot & 
                 & \top & \bot \\
                 &  & \bot & \top & \bot &  & \bot & \bot & 
                 & \top & \bot \\
                \midrule
                 &  & 1 & 2 & 1 &  & 3 & 1 & 
                 & \mathbf 4 & 1 \\
                \bottomrule
            \end{array}
        \]
    \end{nlist}
\end{myproof}

From the proofs above,
one can guess that
every proposition with finite variables
can be determined its truth values by checking all the interpretations.%
\footnote{It can even be programmed to do so;
see \url{https://proofmood.mindconnect.cc/en/TruthTable/tr_table.php}.}

\cref{def:truth} also gives rise to a more general definition.

\begin{definition}[Models]
    \label{def:model}
    Given a collection \(\mathcal A=\{A_1,\dots,A_n\}\) of propositions,
    if each \(A_i\) are true under an interpretation \(I\),
    then we say that
    \(I\) \textdef{satisfies} the collection \(\mathcal A\).

    Now if an interpretation \(I\) satisfies a proposition \(A\),
    then \(I\) is called a \define{model} of \(A\),
    and is denoted by \(I\denote{\models} A\)
    (read as ``\(I\) models \(A\)'').
    Similarly, for a collection \(\mathcal A\) of propositions,
    we define \(I\models\mathcal A\)
    if \(I\) models each proposition in \(\mathcal A\).

    If a proposition (or a collection of propositions)
    has an interpretation which satisfies it,
    then it is said to be \define{consistent}.
    If not, then it is said to be \define{inconsistent}
    or a \define{contradiction}.
\end{definition}

Why are models important?
One possible guess is that
under these models
our world of propositions become richer in information.
However,
for an arbitrarily chosen proposition,
there might not exist any models of it;
hence we should always check its existence.
The following example demonstrates a simple instance of this.

\begin{example}
    \label{exm:satisfy}
    Given a consistent proposition (but not a tautology) \(A\),
    consider the collection \(\Phi=\{A,\neg A\}\).
    Then it is clear that \(\neg A\) is also consistent.
    However, there is no interpretation which satisfies \(\Phi\).
\end{example}

Thus a collection of consistent propositions
need not be consistent by itself.
In contrast, it is clear that
a collection of inconsistent propositions
is always inconsistent.

Having defined models,
it is natural to think of collections of models.

\begin{definition}[Model Sets]
    \label{def:modelset}
    Given a proposition \(A\),
    we denote the set of all models of \(A\) by \(\denote{\Mod} A\).
    Then for a collection \(\Phi=\indexset{A_i}{i=1}{n}\)
    of propositions,
    the set of all models \(\Mod\Phi\) is clearly the set
    \[
        \bigcap_{i=1}^n\Mod A_i
    \]
    from \cref{def:model}.
    The sets \(\Mod A\) and \(\Mod\Phi\) are called
    \define[model set]{model sets} of \(A\) and \(\Phi\), respectively.
\end{definition}

Note that set-theoretic notions are used here;
for notations one may be encouraged to read \cref{sec:sets} first.
Since definitions of intersections and unions rely only on connectives,
it is not a circular reasoning.

\begin{remark}
    \label{rem:modelset}
    Let \(A\), \(A_1,\dots,A_n\), \(B\) be propositions,
    and \(\Phi\), \(\Psi\) be collections of propositions.
    Also let \(\mathcal I\) be the set of all possible interpretations.
    Then the followings hold:
    \begin{itemize}
        \item \(\Mod(A_1,\dots,A_n)=\Mod(A_1\wedge\dots\wedge A_n)\);
        \item \(A\) is a tautology if and only if \(\Mod A=\mathcal I\);
        \item \(A\Iff B\) if and only if \(\Mod A=\Mod B\);
        \item \(A\) (or \(\Phi\)) is consistent
        if and only if \(\Mod A\neq\varnothing\)
        (or \(\Mod\Phi\neq\varnothing\));
        \item \(A\) (or \(\Phi\)) is inconsistent
        if and only if \(\Mod A=\varnothing\)
        (or \(\Mod\Phi=\varnothing\));
        \item If \(\Phi\subset\Psi\), then \(\Mod\Phi\supset\Mod\Psi\).
    \end{itemize}
\end{remark}

\begin{definition}[Consequences]
    \label{def:consequence}
    Given a collection \(\Phi\) of propositions
    and another proposition \(B\),
    \(B\) is said to be
    a \define{logical consequence} (or consequence in short)
    if \(\Mod\Phi\subset\Mod B\),
    and is denoted by \(\Phi\models B\).
    If \(\Phi=\varnothing\),
    then we denote as \(\models B\)
    rather than \(\varnothing\models B\).
\end{definition}

If \(\Phi\) in the above is empty,
then the sentence \(\models B\) means that
\(B\) is true under every interpretation which satisfies \(\Phi\);
but since \(\Phi\) is empty,
\(B\) is true under any interpretation.%
\footnote{If one wonders why this is true,
see \cref{sec:proof}.}
Thus the notation \(\models B\) implies that \(B\) is a tautology.

Note that the symbol \(\models\) is abused
in \cref{def:model,def:consequence}.
In the latter case,
\(\Phi\models B\) is read as ``\(\Phi\) entails \(B\).''

\begin{example}
    \label{prop:consq}
    The propositions obtained
    by substituting \(\models\)'s in place of \(\iff\)
    in \cref{exm:taut}
    are true.
    Moreover,
    for propositions \(P\), \(Q\), \(R\) and \(S\),
    the followings hold.
    \begin{nlist}
        \item \(P,P\to Q\models Q\).
        \item \(P\to Q,\neg Q\models \neg P\).
        \item \(P\to Q,Q\to R\models P\to R\).
        \item \(P\vee Q,\neg Q\models P\).
        \item \(P\to R,Q\to S,P\vee Q\models R\vee S\).
        \item \(P\to R,Q\to S,\neg R\vee\neg S\models\neg P\vee\neg Q\).
        \item \(P\wedge Q\models P\).
        \item \(P\models P\vee Q\).
        \item \(P\to Q, P\to R\models P\to(Q\wedge R)\).
        \item \(\models P\vee\neg P\).
        \item \(Q\models P\to Q\).
    \end{nlist}
\end{example}
\begin{myproof}
    First,
    consider a tautology from \cref{exm:taut}
    in the form of \(X\iff Y\).
    Then if an interpretation \(I\) models \(X\),
    the fact that \(I\) models \(X\iff Y\) implies that
    \(I\) models \(Y\);
    thus \(X\models Y\).

    Now we prove the additional items.

    \begin{nlist}
        \item Let \(I\) be an interpretation
        which models both \(P\) and \(P\to Q\).
        Then by the truth table
        \[
            \begin{array}{c c}
                \toprule
                P & Q \\
                \midrule
                \top & \top \\
                \top & \bot \\
                \bot & \top \\
                \bot & \bot \\
                \midrule
                \multicolumn{2}{c}{\text{Hierarchy}} \\
                \bottomrule
            \end{array}
            \;
            \begin{array}{c c c c c c c}
                \toprule
                P & \wedge & ( & P & \to & Q & ) \\
                \midrule
                \top & \top &  & \top & \top & \top &  \\
                \top & \bot &  & \top & \bot & \bot &  \\
                \bot & \bot &  & \bot & \top & \top &  \\
                \bot & \bot &  & \bot & \top & \bot &  \\
                \midrule
                1 & \mathbf 3 &  & 1 & 2 & 1  & \\
                \bottomrule
            \end{array}
        \]
        we observe that
        \(I\) assigns the truth value \(\top\) to \(Q\).

        \item Let \(I\) be an interpretation
        which models both \(P\to Q\) and \(\neg Q\).
        Then by the truth table
        \[
            \begin{array}{c c}
                \toprule
                P & Q \\
                \midrule
                \top & \top \\
                \top & \bot \\
                \bot & \top \\
                \bot & \bot \\
                \midrule
                \multicolumn{2}{c}{\text{Hierarchy}} \\
                \bottomrule
            \end{array}
            \;
            \begin{array}{c c c c c c c c}
                \toprule
                ( & P & \to & Q & ) & \wedge & \neg & Q \\
                \midrule
                 & \top & \top & \top &  & \bot & \bot & \top \\
                 & \top & \bot & \bot &  & \bot & \top & \bot \\
                 & \bot & \top & \top &  & \bot & \bot & \top \\
                 & \bot & \top & \bot &  & \top & \top & \bot \\
                \midrule
                 & 1 & 2 & 1 &  & \mathbf 3 & 2 & 1 \\
                \bottomrule
            \end{array}
        \]
        we observe that
        \(I\) assigns the truth value \(\bot\) to \(P\);
        hence \(I\) models \(\neg P\).

        \item Let \(I\) be an interpretation
        which models both \(P\to Q\) and \(Q\to R\).
        Then by the truth table
        \[
            \begin{array}{c c c}
                \toprule
                P & Q & R \\
                \midrule
                \top & \top & \top \\
                \top & \top & \bot \\
                \top & \bot & \top \\
                \top & \bot & \bot \\
                \bot & \top & \top \\
                \bot & \top & \bot \\
                \bot & \bot & \top \\
                \bot & \bot & \bot \\
                \midrule
                \multicolumn{3}{c}{\text{Hierarchy}} \\
                \bottomrule
            \end{array}
            \;
            \begin{array}{c c c c c c c c c c c}
                \toprule
                ( & P & \to & Q & ) & \wedge & ( & Q & \to & R & ) \\
                \midrule
                 & \top & \top & \top & 
                 & \top &  & \top & \top & \top &  \\
                 & \top & \top & \top & 
                 & \bot &  & \top & \bot & \bot &  \\
                 & \top & \bot & \bot & 
                 & \bot &  & \bot & \top & \top &  \\
                 & \top & \bot & \bot & 
                 & \bot &  & \bot & \top & \bot &  \\
                 & \bot & \top & \top & 
                 & \top &  & \top & \top & \top &  \\
                 & \bot & \top & \top & 
                 & \bot &  & \top & \bot & \bot &  \\
                 & \bot & \top & \bot & 
                 & \top &  & \bot & \top & \top &  \\
                 & \bot & \top & \bot & 
                 & \top &  & \bot & \top & \bot &  \\
                \midrule
                 & 1 & 2 & 1 &  & \mathbf 3 &  & 1 & 2 & 1  & \\
                \bottomrule
            \end{array}
            \qquad
            \begin{array}{c c}
                \toprule
                P & R \\
                \midrule
                \top & \top \\
                \top & \bot \\
                \bot & \top \\
                \bot & \bot \\
                \midrule
                \multicolumn{2}{c}{\text{Hierarchy}} \\
                \bottomrule
            \end{array}
            \;
            \begin{array}{c c c}
                \toprule
                P & \to & R \\
                \midrule
                \top & \top & \top \\
                \top & \bot & \bot \\
                \bot & \top & \top \\
                \bot & \top & \bot \\
                \midrule
                1 & \mathbf 2 & 1 \\
                \bottomrule
            \end{array}
        \]
        we observe that \(I\) always models \(P\to R\).
        
        \item Let \(I\) be an interpretation which models
        both \(P\vee Q\) and \(\neg Q\).
        Then by the truth table
        \[
            \begin{array}{c c}
                \toprule
                P & Q \\
                \midrule
                \top & \top \\
                \top & \bot \\
                \bot & \top \\
                \bot & \bot \\
                \midrule
                \multicolumn{2}{c}{\text{Hierarchy}} \\
                \bottomrule
            \end{array}
            \;
            \begin{array}{c c c c c c c c}
                \toprule
                ( & P & \vee & Q & ) & \wedge & \neg & Q \\
                \midrule
                 & \top & \top & \top &  & \bot & \bot & \top \\
                 & \top & \top & \bot &  & \top & \top & \bot \\
                 & \bot & \top & \top &  & \bot & \bot & \top \\
                 & \bot & \bot & \bot &  & \bot & \top & \bot \\
                \midrule
                 & 1 & 2 & 1 &  & \mathbf 3 & 2 & 1 \\
                \bottomrule
            \end{array}
        \]
        we observe that \(I\) always models \(P\).

        \item Let \(I\) be an interpretation which models
        \(P\to R\), \(Q\to S\) and \(P\vee Q\).
        Then by the truth table
        \[
            \begin{array}{c c c c}
                \toprule
                P & Q & R & S \\
                \midrule
                \top & \top & \top & \top \\
                \top & \top & \top & \bot \\
                \top & \top & \bot & \top \\
                \top & \top & \bot & \bot \\
                \top & \bot & \top & \top \\
                \top & \bot & \top & \bot \\
                \top & \bot & \bot & \top \\
                \top & \bot & \bot & \bot \\
                \bot & \top & \top & \top \\
                \bot & \top & \top & \bot \\
                \bot & \top & \bot & \top \\
                \bot & \top & \bot & \bot \\
                \bot & \bot & \top & \top \\
                \bot & \bot & \top & \bot \\
                \bot & \bot & \bot & \top \\
                \bot & \bot & \bot & \bot \\
                \midrule
                \multicolumn{4}{c}{\text{Hierarchy}} \\
                \bottomrule
            \end{array}
            \;
            \begin{array}{c c c c c c c c c c c c c c c c c}
                \toprule
                ( & P & \to & R & )
                & \wedge & ( & Q & \to & S & )
                & \wedge & ( & P & \vee & Q & ) \\
                \midrule
                 & \top & \top & \top & 
                 & \top &  & \top & \top & \top & 
                 & \top &  & \top & \top & \top &  \\
                 & \top & \top & \top & 
                 & \bot &  & \top & \bot & \bot & 
                 & \bot &  & \top & \top & \top &  \\
                 & \top & \bot & \bot & 
                 & \bot &  & \top & \top & \top & 
                 & \bot &  & \top & \top & \top &  \\
                 & \top & \bot & \bot & 
                 & \bot &  & \top & \bot & \bot & 
                 & \bot &  & \top & \top & \top &  \\
                 & \top & \top & \top & 
                 & \top &  & \bot & \top & \top & 
                 & \top &  & \top & \top & \bot &  \\
                 & \top & \top & \top & 
                 & \top &  & \bot & \top & \bot & 
                 & \top &  & \top & \top & \bot &  \\
                 & \top & \bot & \bot & 
                 & \bot &  & \bot & \top & \top & 
                 & \bot &  & \top & \top & \bot &  \\
                 & \top & \bot & \bot & 
                 & \bot &  & \bot & \top & \bot & 
                 & \bot &  & \top & \top & \bot &  \\
                 & \bot & \top & \top & 
                 & \top &  & \top & \top & \top & 
                 & \top &  & \bot & \top & \top &  \\
                 & \bot & \top & \top & 
                 & \bot &  & \top & \bot & \bot & 
                 & \bot &  & \bot & \top & \top &  \\
                 & \bot & \top & \bot & 
                 & \top &  & \top & \top & \top & 
                 & \top &  & \bot & \top & \top &  \\
                 & \bot & \top & \bot & 
                 & \bot &  & \top & \bot & \bot & 
                 & \bot &  & \bot & \top & \top &  \\
                 & \bot & \top & \top & 
                 & \top &  & \bot & \top & \top & 
                 & \bot &  & \bot & \bot & \bot &  \\
                 & \bot & \top & \top & 
                 & \top &  & \bot & \top & \bot & 
                 & \bot &  & \bot & \bot & \bot &  \\
                 & \bot & \top & \bot & 
                 & \top &  & \bot & \top & \top & 
                 & \bot &  & \bot & \bot & \bot &  \\
                 & \bot & \top & \bot & 
                 & \top &  & \bot & \top & \bot & 
                 & \bot &  & \bot & \bot & \bot &  \\
                \midrule
                 & 1 & 2 & 1 & 
                 & 3 &  & 1 & 2 & 1 & 
                 & \mathbf 4 &  & 1 & 2 & 1  & \\
                \bottomrule
            \end{array}
        \]
        we observe that
        \(I\) always models \(R\vee S\).
        
        \item Let \(I\) be an interpretation which models
        \(P\to R\), \(Q\to S\) and \(\neg R\vee\neg S\).
        Then by \cref{exm:taut}(7)
        we have \(\neg R\vee\neg S\Iff\neg(R\wedge S)\),
        and from the truth table
        \[
            \begin{array}{c c c c}
                \toprule
                P & Q & R & S \\
                \midrule
                \top & \top & \top & \top \\
                \top & \top & \top & \bot \\
                \top & \top & \bot & \top \\
                \top & \top & \bot & \bot \\
                \top & \bot & \top & \top \\
                \top & \bot & \top & \bot \\
                \top & \bot & \bot & \top \\
                \top & \bot & \bot & \bot \\
                \bot & \top & \top & \top \\
                \bot & \top & \top & \bot \\
                \bot & \top & \bot & \top \\
                \bot & \top & \bot & \bot \\
                \bot & \bot & \top & \top \\
                \bot & \bot & \top & \bot \\
                \bot & \bot & \bot & \top \\
                \bot & \bot & \bot & \bot \\
                \midrule
                \multicolumn{4}{c}{\text{Hierarchy}} \\
                \bottomrule
            \end{array}
            \;
            \begin{array}{c c c c c c c c c c c c c c c c c c}
                \toprule
                ( & P & \to & R & )
                & \wedge & ( & Q & \to & S & )
                & \wedge & \neg & ( & R & \wedge & S & ) \\
                \midrule
                 & \top & \top & \top & 
                 & \top &  & \top & \top & \top & 
                 & \bot & \bot &  & \top & \top & \top &  \\
                 & \top & \top & \top & 
                 & \bot &  & \top & \bot & \bot & 
                 & \bot & \top &  & \top & \bot & \bot &  \\
                 & \top & \bot & \bot & 
                 & \bot &  & \top & \top & \top & 
                 & \bot & \top &  & \bot & \bot & \top &  \\
                 & \top & \bot & \bot & 
                 & \bot &  & \top & \bot & \bot & 
                 & \bot & \top &  & \bot & \bot & \bot &  \\
                 & \top & \top & \top & 
                 & \top &  & \bot & \top & \top & 
                 & \bot & \bot &  & \top & \top & \top &  \\
                 & \top & \top & \top & 
                 & \top &  & \bot & \top & \bot & 
                 & \top & \top &  & \top & \bot & \bot &  \\
                 & \top & \bot & \bot & 
                 & \bot &  & \bot & \top & \top & 
                 & \bot & \top &  & \bot & \bot & \top &  \\
                 & \top & \bot & \bot & 
                 & \bot &  & \bot & \top & \bot & 
                 & \bot & \top &  & \bot & \bot & \bot &  \\
                 & \bot & \top & \top & 
                 & \top &  & \top & \top & \top & 
                 & \bot & \bot &  & \top & \top & \top &  \\
                 & \bot & \top & \top & 
                 & \bot &  & \top & \bot & \bot & 
                 & \bot & \top &  & \top & \bot & \bot &  \\
                 & \bot & \top & \bot & 
                 & \top &  & \top & \top & \top & 
                 & \top & \top &  & \bot & \bot & \top &  \\
                 & \bot & \top & \bot & 
                 & \bot &  & \top & \bot & \bot & 
                 & \bot & \top &  & \bot & \bot & \bot &  \\
                 & \bot & \top & \top & 
                 & \top &  & \bot & \top & \top & 
                 & \bot & \bot &  & \top & \top & \top &  \\
                 & \bot & \top & \top & 
                 & \top &  & \bot & \top & \bot & 
                 & \top & \top &  & \top & \bot & \bot &  \\
                 & \bot & \top & \bot & 
                 & \top &  & \bot & \top & \top & 
                 & \top & \top &  & \bot & \bot & \top &  \\
                 & \bot & \top & \bot & 
                 & \top &  & \bot & \top & \bot & 
                 & \top & \top &  & \bot & \bot & \bot &  \\
                \midrule
                 & 1 & 2 & 1 & 
                 & 3 &  & 1 & 2 & 1 & 
                 & \mathbf 4 & 3 &  & 1 & 2 & 1  & \\
                \bottomrule
            \end{array}
        \]
        we observe that
        \(I\) always models \(\neg P\vee\neg Q\).
        
        \item Let \(I\) be an interpretation which models \(P\wedge Q\).
        Then by the truth table
        \[
            \begin{array}{c c}
                \toprule
                P & Q \\
                \midrule
                \top & \top \\
                \top & \bot \\
                \bot & \top \\
                \bot & \bot \\
                \midrule
                \multicolumn{2}{c}{\text{Hierarchy}} \\
                \bottomrule
            \end{array}
            \;
            \begin{array}{c c c}
                \toprule
                P & \wedge & Q \\
                \midrule
                \top & \top & \top \\
                \top & \bot & \bot \\
                \bot & \bot & \top \\
                \bot & \bot & \bot \\
                \midrule
                1 & \mathbf 2 & 1 \\
                \bottomrule
            \end{array}
        \]
        we observe that
        \(I\) always models \(P\).
        
        \item Let \(I\) be an interpretation that models \(P\).
        Then it is clear that
        \(I\) always models \(P\vee Q\).
        
        \item Let \(I\) be an interpretation that models
        both \(P\to Q\) and \(P\to R\).
        Then from the truth table
        \[
            \begin{array}{c c c}
                \toprule
                P & Q & R \\
                \midrule
                \top & \top & \top \\
                \top & \top & \bot \\
                \top & \bot & \top \\
                \top & \bot & \bot \\
                \bot & \top & \top \\
                \bot & \top & \bot \\
                \bot & \bot & \top \\
                \bot & \bot & \bot \\
                \midrule
                \multicolumn{3}{c}{\text{Hierarchy}} \\
                \bottomrule
            \end{array}
            \;
            \begin{array}{c c c c c c c c c c c}
                \toprule
                ( & P & \to & Q & )
                & \wedge & ( & P & \to & R & ) \\
                \midrule
                 & \top & \top & \top & 
                 & \top &  & \top & \top & \top &  \\
                 & \top & \top & \top & 
                 & \bot &  & \top & \bot & \bot &  \\
                 & \top & \bot & \bot & 
                 & \bot &  & \top & \top & \top &  \\
                 & \top & \bot & \bot & 
                 & \bot &  & \top & \bot & \bot &  \\
                 & \bot & \top & \top & 
                 & \top &  & \bot & \top & \top &  \\
                 & \bot & \top & \top & 
                 & \top &  & \bot & \top & \bot &  \\
                 & \bot & \top & \bot & 
                 & \top &  & \bot & \top & \top &  \\
                 & \bot & \top & \bot & 
                 & \top &  & \bot & \top & \bot &  \\
                \midrule
                 & 1 & 2 & 1 & 
                 & \mathbf 3 &  & 1 & 2 & 1  & \\
                \bottomrule
            \end{array}
            \;
            \begin{array}{c c c c c c c}
                \toprule
                P & \to & ( & Q & \wedge & R & ) \\
                \midrule
                \top & \top &  & \top & \top & \top &  \\
                \top & \bot &  & \top & \bot & \bot &  \\
                \top & \bot &  & \bot & \bot & \top &  \\
                \top & \bot &  & \bot & \bot & \bot &  \\
                \bot & \top &  & \top & \top & \top &  \\
                \bot & \top &  & \top & \bot & \bot &  \\
                \bot & \top &  & \bot & \bot & \top &  \\
                \bot & \top &  & \bot & \bot & \bot &  \\
                \midrule
                1 & \mathbf 3 &  & 1 & 2 & 1  & \\
                \bottomrule
            \end{array}
        \]
        we observe that
        \(I\) always models \(P\to(Q\wedge R)\).
        
        \item Let \(I\) be an interpretation that models
        an empty set of propositions;
        that is, let \(I\) be any interpretation.
        Then the truth table
        \[
            \begin{array}{c}
                \toprule
                P \\
                \midrule
                \top \\
                \bot \\
                \midrule
                \multicolumn{1}{c}{\text{Hierarchy}} \\
                \bottomrule
            \end{array}
            \;
            \begin{array}{c c c c}
                \toprule
                P & \vee & \neg & P \\
                \midrule
                \top & \top & \bot & \top \\
                \bot & \top & \top & \bot \\
                \midrule
                1 & \mathbf 3 & 2 & 1 \\
                \bottomrule
            \end{array}
        \]
        we observe that
        \(I\) always models \(P\vee\neg P\).

        \item Let \(I\) be an interpretation that models \(Q\).
        Then from the truth table
        \[
            \begin{array}{c c}
                \toprule
                P & Q \\
                \midrule
                \top & \top \\
                \top & \bot \\
                \bot & \top \\
                \bot & \bot \\
                \midrule
                \multicolumn{2}{c}{\text{Hierarchy}} \\
                \bottomrule
            \end{array}
            \;
            \begin{array}{c c c}
                \toprule
                P & \to & Q \\
                \midrule
                \top & \top & \top \\
                \top & \bot & \bot \\
                \bot & \top & \top \\
                \bot & \top & \bot \\
                \midrule
                1 & \mathbf 2 & 1 \\
                \bottomrule
            \end{array}
        \]
        we observe that
        \(P\to Q\) is true whenever \(Q\) is true.
        Hence \(I\) models \(P\to Q\).
        \rightqed
    \end{nlist}
\end{myproof}

Before we move on to the next section,
we state a very important result.

\begin{lemma}
    \label{lem:consq}
    For any propositions \(A_1,\dots,A_n\) and \(B\),
    \begin{nlist}
        \item \(A_1,\dots,A_n\models B\)
        if and only if
        \(A_1\wedge\cdots\wedge A_n\to B\) is a tautology.
        \item \(A_1,\dots,A_n\models B\)
        if and only if
        \(A_1,\dots,A_{n-1}\models A_n\to B\).
        \item \(A_1,\dots,A_n\models B\)
        if and only if
        \(\{A_1,\dots,A_n,\neg B\}\) is inconsistent.
    \end{nlist}
\end{lemma}
\begin{sketch}
    Since we do not know the number of variables,
    it is impossible (and impractical for large \(n\))
    to draw the full truth table.
    Hence we need to reason them by words.
    ``Proving'' something is to be discussed in detail later;
    for now, we present a tangible (but not yet justified) solution.
    To prove the statement ``\(A\) if and only if \(B\),''
    we need to prove two statements
    ``\(A\) implies \(B\)''
    and ``\(B\) implies \(A\).''
\end{sketch}
\begin{myproof}
    Let \(A_1,\dots,A_n\) and \(B\) be propositions.
    \begin{nlist}
        \item First, suppose that \(A_1,\dots,A_n\) entails \(B\),
        and let \(I\) be an interpretation.
        
        If \(A_1\wedge\dots\wedge A_n\) is false under \(I\),
        then the proposition \(A_1\wedge\dots\wedge A_n\to B\)
        is vacuously true.
        
        Now suppose that
        \(A_1\wedge\dots\wedge A_n\) is true under \(I\).
        Then each \(A_1,\dots,A_n\) is true under \(I\)
        and so \(I\in\Mod(A_1,\dots,A_n)\).
        Since \(A_1,\dots,A_n\models B\) implies
        \(\Mod(A_1,\dots,A_n)\subset\Mod B\),
        \(I\in\Mod B\) and \(B\) is also true under \(I\).

        Therefore the statement \(A_1\wedge\dots\wedge A_n\to B\)
        is always true under any interpretation,
        and thus is a tautology.

        Conversely,
        suppose that \(A_1\wedge\dots\wedge A_n\to B\) is a tautology,
        and let \(I\in\Mod(A_1,\dots,A_n)\).
        Then since each \(A_1,\dots,A_n\) is true under \(I\),
        \(A_1\wedge\dots\wedge A_n\) is also true under \(I\).
        Now by our assumption,
        \(A_1\wedge\dots\wedge A_n\to B\) must be true under \(I\);
        thus we have that
        \(B\) is true under \(I\), that is, \(I\in\Mod B\).

        Therefore \(\Mod(A_1,\dots,A_n)\subset\Mod B\),
        and we conclude that \(\{A_1,\dots,A_n\}\) entails \(B\).

        \item First, suppose that \(A_1,\dots,A_n\) entails \(B\),
        and let \(I\in\Mod(A_1,\dots,A_{n-1})\).

        If \(A_n\) is false under \(I\),
        then the proposition \(A_n\to B\) is vacuously true under \(I\).

        Now if \(A_n\) is true under \(I\),
        then \(I\in\Mod(A_1,\dots,A_n)\);
        since \(A_1,\dots,A_n\models B\),
        \(I\in\Mod B\)
        and the proposition \(A_n\to B\) is also true under \(I\).

        Therefore \(I\in\Mod B\) in either cases
        and \(\Mod(A_1,\dots,A_{n-1})\subset\Mod(A_n\to B)\) holds;
        thus we conclude that
        \(\{A_1,\dots,A_{n-1}\}\) entails \(A_n\to B\).

        Conversely, suppose that
        \(A_1,\dots,A_{n-1}\) entails \(A_n\to B\),
        and let \(I\) be in \(\Mod(A_1,\dots,A_n)\).
        Since \(\Mod(A_1,\dots,A_n)\subset\Mod(A_1,\dots,A_{n-1})\)
        by \cref{rem:modelset},
        \(I\in\Mod(A_1,\dots,A_{n-1})\) and thus \(I\in\Mod(A_n\to B)\).
        Note that \(I\in\Mod(A_1,\dots,A_n)\) also models \(A_n\);
        thus \(I\) models \(B\), that is, \(I\in\Mod B\).

        Therefore \(\Mod(A_1,\dots,A_n)\subset\Mod B\),
        and we conculde that \(\{A_1,\dots,A_n\}\) entails \(B\).

        \item First, suppose that \(\{A_1,\dots,A_n\}\) entails \(B\),
        and let \(I\) be an interpretation.
        
        If \(I\) does not model \(\{A_1,\dots,A_n\}\),
        then \(I\) cannot model \(\{A_1,\dots,A_n,\neg B\}\)
        by \cref{rem:modelset}.

        If \(I\) models \(\{A_1,\dots,A_n\}\),
        then the assumption implies that \(I\) also models \(B\).
        But then \(I\) cannot model \(\neg B\)
        and thus \(I\notin\Mod(A_1,\dots,A_n,\neg B)\).

        Hence no interpretation can model \(\{A_1,\dots,A_n,\neg B\}\),
        and thus it is inconsistent.

        To prove the converse,
        we prove its contrapositive.
        Suppose that \(\{A_1,\dots,A_n\}\) does not entail \(B\).
        Then there exists an interpretation \(I\)
        which models \(\{A_1,\dots,A_n\}\) but does not model \(B\).
        Such an interpretationi must model \(\neg B\),
        and therefore \(\{A_1,\dots,A_n,\neg B\}\) is consistent.

        Hence if \(\{A_1,\dots,A_n,\neg B\}\) is inconsistent,
        then \(\{A_1,\dots,A_n\}\) entails \(B\).
        \rightqed
    \end{nlist}
\end{myproof}