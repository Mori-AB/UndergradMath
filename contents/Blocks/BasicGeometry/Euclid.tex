\setsection{Euclidean Geometry}
\label{sec:euclidgeo}

Lets talk about some mathematical history.
There are many groundbreaking works inside mathematics.
I personally argue that
one of the most important incidents of mathematics
was when non-Euclidean geometry was founded in the 19th century.
Mathematicians were freed from the plane that
they have lived on for thousands of years,
and finally started to study the space itself,
rather than shapes lying on that space.

But in order to appreciate the beauty of non-Euclidean geometry,
one must first take a look at what Euclidean geoemtry means.
Euclid decided to have five postulates and some definitions
to describe the geometry he knew.
The definitions are not precise yet;
we just look at the postulates themselves.

\begin{definition}[Euclid's Postulates]
    \label{def:euclidpos}
    The postulates of Euclidean geometry are the following:
    \begin{nlist}
        \item Given any two points,
        there uniquely exists a line connecting the two points.
        
        \item Given any two segments \(AB\) and \(CD\),
        there exists a point \(E\) such that
        \(B\) is in between \(A\) and \(E\)
        and the two segments \(EB\) and \(CD\) have the same length.
        
        \item Given any two points \(O\) and \(A\),
        there exists a circle centered at \(O\) with radius \(OA\).
        
        \item Every right angle is equal to one another.
        
        \item If a straight line falls on two straight lines
        making two interior angles on the same side
        which are less than two right angles,
        then the two straight lines meet on that side of the plane.
    \end{nlist}
\end{definition}

One can almost immediately see the difference in the last postulate.
For example, lets say that
we are to draw objects with a ruler and a compass
in our everyday lives.
Then it is quite clear that the first postulate can be achieved;
just lay the ruler on both points and draw a line.
Similarly, the second postulate can be achieved with our everyday ruler,
since we can measure and copy the distance.
The third postulate is achieved by using a compass,
and the fourth is achieved
by defining the angle of a straight line to be \(180^\circ\).
Then by the definition of right angles,
every right angle has a size of \(90^\circ\) and we are done.

However, the last postulate of \cref{def:euclidpos},
called the \define{parallel postulate},
just seems less trivial.
How can one guarantee that the two straight lines meet?
Naturally, many proofs were attempted throughout history.
It is now known that
the parallel postulate is independent of the rest of the postulates.

But after centuries, it turned out that
Euclidean Geometry was flawed even without the parallel postulate.
For example,
consider the following proof of constructing an equilateral triangle
which has the given segment as one of its sides.

\begin{example}
    \label{exm:equitri}
    Let the line segment \(AB\) be given.
    Applying the third postulate twice
    we obtain two circles;
    one centered at \(A\) with radius \(AB\),
    and the other centered at \(B\) with radius \(BA\).
    Then the two circles meet at a point;
    denote it by \(C\).
    
    Now the segments \(AC\) and \(AB\) have equal length
    since they are both radii of a circle.
    Also, the segments \(AB\) and \(BC\) have equal length
    since they are both radii of a circle.
    Thus the three segments \(AC\), \(AB\) and \(BC\) have equal length
    and thus the triangle \(ABC\) is an equilateral triangle.
\end{example}

One might call this proof flawless,
but there is one small minor detail:
how do you guarantee that the two circles meet at some point?
Suppose that the space under discussion were \(\Zmath^2\)
with \(A(0,0)\) and \(B(1,0)\).
Then the circles centered each at \(A\) and \(B\) with radius 1
clearly do not meet at any point in \(\Zmath^2\).
Is this nitpicking?
Actually, it's not.
Since we did not define what plane should the discussion be,
this is still a valid counterexample.

After this, some mathematicians established new axioms
to refine the classical Euclidean geometry.
One of the most popular axioms are from
\textcite{Hil05},
who refined the explicit postulates and implicit axioms by Euclid
as five groups of axioms.
We wish to follow Hilbert's axioms
and see how it improves Euclid's geometry.
Only after we have established Euclidean geometry
we are able to see examples of non-Euclidean geometry.