\setsection{Vector Spaces}
\label{sec:vs}

In algebra, one defines objects
---
one might say, ``categories'' of objects
---
by some properties that we wish them to have.
We already have seen this kind of definitions,
particularly the notion of fields throughout \cref{chap:numbers}.
We now present a new kind of object called ``vector spaces.''

\begin{definition}
    \label{def:vs}
    Let \(V\) be a set
    and \(F\) be a field (for example, \(\Qmath\) or \(\Rmath\)).
    Now let
    \[
        +:V\times V\to V
        \qquad \text{and} \qquad
        \cdot:F\times V\to V
    \]
    be binary operations defined from \(V\) and \(F\).
    These are each called
    \textdef{addition} and \textdef{scalar multiplication} of \(V\),
    and are written as
    \[
        +(v,w)=v+w
        \qquad \text{and} \qquad
        \cdot(a,v)=av.
    \]
    
    Suppose that
    the operations \(+\) and \(\cdot\) satisfy the following conditions:
    \begin{enumerate}[label=(V-\arabic*)]
        \item For any \(u,v,w\in V\),
        \((u+v)+w=u+(v+w)\).
        \hfill(Associative addition)
        
        \item For any \(u,v\in V\),
        \(u+v=v+u\).
        \hfill(Commutative addition)

        \item There exists \(\nu\in V\) such that
        \(v+\nu=v\) for all \(v\in V\).
        \hfill(Additive identity, zero vector)

        \item Given any \(v\in V\),
        there exists \(\tilde v\in V\) such that
        \(v+\tilde v=\nu\).
        \hfill(Additive inverse)

        \item For any \(v\in V\) and \(a,b\in F\),
        \((a+b)v=av+bv\)
        \hfill(Scalar distribution)

        \item For any \(u,v\in V\) and \(a\in F\),
        \(a(u+v)=au+av\)
        \hfill(Vector distribution)

        \item For any \(v\in V\) and \(a,b\in F\),
        \(a(bv)=(ab)v=b(av)\).
        \hfill(Compatibility)

        \item For any \(v\in V\),
        \(1v=v\).
    \end{enumerate}
    Then the tuple \((V,+,\cdot)\) is called
    a \define[vector space]{vector space over \(F\)},
    and the members of \(V\) and \(F\) are called
    \define{vectors} and \define{scalars}, respectively.
    If the scalar field is clear from the context,
    we simply say that \(V\) is a vector space.
\end{definition}

Since \(\nu\) and \(\tilde v\) play exactly the roles of
\(0\) and \(-v\) in \(\Rmath\),
we simply denote \(\nu\) as \(0\) and \(\tilde v\) as \(-v\).
Thus the conditions (V-3) and (V-4) are rewritten as
\begin{enumerate}[label=(V-\arabic*)]
    \setcounter{enumi}{2}
    \item There exists \(0\in V\) such that
    \(v+0=v\) for all \(v\in V\).

    \item Given any \(v\in V\),
    there exists \(-v\in V\) such that \(v+(-v)=0\).
\end{enumerate}

But are identities and inverses unique?
The answer is positive.
In fact, the same argument will show up many times throughout algebra.

\begin{proposition}
    \label{prop:vsidinvuniq}
    Let \(V\) be a vector space,
    and \(v\in V\).
    Then the zero vector \(0\)
    and the additive inverse \(-v\) are unique.
\end{proposition}
\begin{myproof}
    First suppose that there are two zero vectors \(0\) and \(0'\).
    Then applying the definition we have
    \[
        0=0+0'=0',
    \]
    thus the zero vector is unique.

    Similarly, suppose that there are two vectors \(u,w\)
    such that \(u+v=0\) and \(v+w=0\).
    Then
    \[
        u=u+0=u+(v+w)=(u+v)+w=0+w=w,
    \]
    thus the additive inverse in (V-4) is unique.
\end{myproof}

Of course, it is common practice to abbreviate addition by inverse.

\begin{convention}
    \label{cnv:vecsubtraction}
    Given a vector space \(V\) and \(v,w\in V\),
    we write the sum \(v+(-w)\) as \(v-w\).
\end{convention}

Note that
we use the same notation for the scalar \(0\) and the vector \(0\).
Context is required to differentiate between the two.
However, the next observation tells us that
it is not much of a trouble.

\begin{observation}
    \label{obv:zerovector}
    Given a vector space \(V\) over \(F\),
    the following holds.
    \begin{nlist}
        \item For any \(v\in V\),
        \(0v=0\).
        \item For any \(a\in F\),
        \(a0=0\).
        \item For any \(v\in V\) and \(a\in F\),
        \(a(-v)=-av=(-a)v\).
        \item Given \(v\in V\) and \(a\in F\),
        \(av=0\) if and only if \(a=0\) or \(v=0\).
    \end{nlist}
\end{observation}
\begin{sketch}
    Since we only have the definition of vector spaces,
    we should try to deduce these from the definitions.
\end{sketch}
\begin{myproof}
    \begin{nlist}
        \item Since \(0=0+0\) in \(F\),
        \[
            0v=(0+0)v=0v+0v
        \]
        implies that \(0v=0\).
    \end{nlist}
\end{myproof}