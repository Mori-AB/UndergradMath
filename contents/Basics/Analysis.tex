\setchapter{Basic Analysis} \label{chap:PMA}

Here we revisit the basic formulations of everyday numbers.
The need for analysis were said to be born
from the observation of \textdef{Fourier series};
which was used to model heat dissipation in a material.
It was quite controversial at the time,
since the notion of convergence were quite different from now.
A number of mathematicians including
Cauchy,
Weierstrass,
Borel,
formulated differential calculus into what is called analysis today.

As shown in its construction,
the completeness property of the reals are essential
in characterizing those,
and it is only natural to think that
most of its important properties come from it.
Only then we are able to proceed to inspect key ideas such as
limits,
continuity,
derivatives and integrals.

Some suffer to accept new definitions and theorems
that seem useless at the time;
but as the history of analysis shows,
these are tools to refine what seemed somewhat ambiguous in calculus.
Hence even though the definitions seem far-fetched,
it will result in theorems that are well-known.

\subimport{Analysis/}{Reals.tex}
\subimport{Analysis/}{MetricSpaces.tex}
\subimport{Analysis/}{Sequences.tex}
\subimport{Analysis/}{CptCnt.tex}