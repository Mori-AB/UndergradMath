\setsection{Metric Spaces} \label{sec:metric}

As mentioned in the start of this chapter,
a primary goal of basic analysis
is to rigorously redefine the concept of limits.
In the casual sense,
the concept requires some notion of closeness.
Here we define closesness by measuring the distance between two points.

Now a question arises:
how do we define distance?
As in defining real numbers,
we first list some properties that are desirable for distances.
Since our intuition is based on \(\Rmath^n\),
these strongly resemble our everyday notion of distance.

\begin{itemize}
    \item Distance is always nonnegative;
    if the distance between two points are zero,
    then the two points must be equal and vice versa.
    
    \item The distance between points \(a\) and \(b\)
    is equal when \(a\) and \(b\) are interchanged;
    that is,
    the order of choosing points is irrelevant.
    
    \item There are no such things
    as ``infinitely far'' or ``infinitely close.''
    
    \item We wish a straight line to be the shortest path
    connecting two points.
\end{itemize}

It is intuitively clear that
the value of a distance should be a number.
The question is: Which number?
First, for the first property,
the number system we use must be ordered.
Also since the third property must hold,
the numbers must be in an Archimedean field.
Knowing that \(\Rmath\) is the largest Archimedean field,
we define the distance to be a (nonnegative) real number.

Hence we have formulated the following definition.

\begin{definition}[Metric (Spaces)]
    \label{def:metric}
    Given a set \(X\),
    the function \(d:X\times X\to\Rmath\) is called
    a \define{metric}
    if it satisfies the following:
    \begin{itemize}
        \item For any \(x,y\in X\), \(d(x,y)\ge 0\).
        Moreover, \(d(x,y)=0\) if and only if \(x=y\).

        \item For any \(x,y\in X\), \(d(x,y)=d(y,x)\).
        
        \item For any \(x,y,z\in X\), \(d(x,z)\le d(x,y)+d(y,z)\).
    \end{itemize}
\end{definition}