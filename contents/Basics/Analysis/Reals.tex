\setsection{The Real Numbers} \label{sec:reals}

% TODO: fix labels when ready
We have constructed the real numbers in \cref{sec:reals}.
Here we take a look at what properties do the real numbers possess,
and what advantages do the formulation have
compared to the `intuitive' definition.

We first restate the defining property of the reals.

\begin{definition}
    \label{def:reals}
    There exists a complete ordered field \(\Rmath\),
    that is, a set with the following properties:
    \begin{itemize}
        \item The set \(\Rmath\) has two (well-known) operations;
        namely, addition and multiplication.
        \item The set \(\Rmath\) is totally ordered.
        \item The set \(\Rmath\) has the least upper bound property;
        given any nonempty set which is bounded above,
        one can find the least upper bound of it in \(\Rmath\).
    \end{itemize}
    The members of \(\Rmath\) are called the \textdef{real numbers}.
\end{definition}

Again, the definition above is well-defined
since any two complete ordered fields are isomorphic.
Hence we do not need to consider how they are constructed,
but only use the defining properties of \(\Rmath\)
(recall that we have constructed \(\Rmath\) in two ways).

The third condition of \cref{def:reals}
is sometimes called the \textdef{completeness} condition.
There are several ways on
calling a set \emph{complete} in real analysis;
it helps to distinguish them later on.

\begin{definition}[Intervals]
    \label{def:interval}
    Given \(a,b\in\Rmath\) with \(a\le b\),
    the \define[interval!open]{open interval} \((a,b)\)
    and \define[interval!closed]{closed interval} \([a,b]\)
    are defined as the sets
    \[
        (a,b)=\set{x\in\Rmath}{a<x<b},
        \qquad
        [a,b]=\set{x\in\Rmath}{a\le x\le b}.
    \]
    Adapting the notation,
    one can also define sets with notation \((a,b]\) and \([a,b)\);
    these are called \define[interval!half-open]{half-open intervals}.
    
    Also, for unbounded sets, we use the notation
    \((-\infty,a)\) and \((a,\infty)\) for sets
    \[
        (-\infty,a)=\set{x\in\Rmath}{x<a},
        \qquad
        (a,\infty)=\set{x\in\Rmath}{a<x}.
    \]
    Notations \((-\infty,a]\) and \([a,\infty)\)
    resemble those of half-open intervals.
    Note that brackets cannot be used for \(\pm\infty\)
    since \(\Rmath\) does not contain an infinitely large number,
    as demonstrated below.
\end{definition}

We first show a seemingly obvious
(and indeed clear from the construction)
fact about the rational numbers:
there are no ``infinitely large'' or ``infinitely small'' values.
\begin{proposition}
    \label{prop:archimedean}
    The following holds in \(\Rmath\):
    \begin{nlist}
        \item The set \(\Nmath\) of natural numbers is unbounded above;
        that is,
        given any \(x>0\), there exists \(n\in\Nmath\) with \(n>x\).
        \item Given any \(x,y\in\Rmath\) with \(x>0\),
        there exists \(n\in\Nmath\) with \(nx>y\).
        \item Given any \(x,y\in\Rmath\),
        there exists a rational number between \(x\) and \(y\)
        (of course, unless \(x=y\)).
    \end{nlist}
\end{proposition}
\begin{myproof}
    \begin{nlist}
        \item This comes directly from the construction of \(\Nmath\),
        or can be viewed as a special result of (2).
        
        \item If \(x>y\), then by taking \(n=1\) we are done.
        
        Now assume that \(x\le y\),
        and consider the set
        \[
            S=\set{mx}{m\in\Nmath, mx\le y}.
        \]
        Then since \(1\in S\),
        \(S\) is clearly nonempty and bounded above by \(y\);
        thus there exists a least upper bound \(z\) of \(S\).
        
        Since \(z-x<z\) is not an upper bound of \(z\),
        there exists some \(nx\in S\) such that \(z-x<nx\le z\).
        Therefore \(nx\le z<(n+1)x\),
        and since \(z\) itself is an upper bound of \(S\),
        we conclude that \((n+1)x\notin S\);
        that is, \((n+1)x>y\).

        \item Without loss of generality,
        we assume that \(x<y\).
        Then since \(y-x>0\),
        by (2) there exists some \(n\in\Nmath\) such that
        \(n(y-x)>1\).

        Applying (2) again we obtain integers \(m_1,m_2\)
        such that \(m_1>nx\) and \(m_2>-nx\).
        Then we observe that \(-m_2<nx<m_1\),
        and thus
        \[
            m-1\le nx<m
        \]
        holds for some integer \(-m_2\le m\le m_1\).
        Hence
        \[
            nx<m\le 1+nx<ny
        \]
        implies 
        \[
            x<\frac{m}{n}<y
        \]
        and we are done.
        \rightqed
    \end{nlist}
\end{myproof}

Note that for \(y>0\), (2) is a direct consequence of (1).
Here we have proved it as in \cite{PMA}
to demonstrate the use of the least upper bound property.
However, \(\Rmath\) is not the only field which (2) holds;
a field which (2) holds is said to be
an \define[field!Archimedean]{Archimedean field}.
For example, \(\Qmath\) is also an Archimedean field.
It is known that
\(\Rmath\) is the largest Archimedean field up to ismorphism.

Item (3) is also stated as
\phrase{the rational numbers are \textdef{dense} in \(\Rmath\)}.
What it exactly means will be investigated in detail
at \cref{chap:PStopo}.

Another famous example of using the least upper bound property
is proving the existence and uniqueness of
the \(n\)-th root of a positive number.

\begin{example}
    \label{exm:nthroot}
    Let \(a>0\) and \(n\) be a positive integer.
    
    Consider the subset \(S=\set{r\in\Qmath}{r^n<a}\) of \(\Rmath\).
    For \(t=a/(1+a)\), \(0<t<1\) implies \(t^n<t<a\).
    Thus \(t\in S\) and so \(S\) is nonempty.
    Also, if \(r>1+a\),
    then
    \[
        r^n>(1+a)^n\ge 1+a>a
    \]
    and thus \(r\notin S\).
    Therefore \(1+a\) is an upper bound for \(S\).

    Now by the least upper bound property of \(\Rmath\),
    there exists a least upper bound \(b=\sup S\).
    Note that \(b\ge a/(1+a)>0\).
    Suppose to the contrary that
    \(b^n\neq a\).
    Then by the trichotomy, either \(b^n<a\) or \(b^n>a\) must hold.

    Suppose that \(b^n<a\).
    Then there exists \(\epsilon>0\) such that
    \[
        \epsilon<\min\left\{1,\frac{a-b^n}{n(b+1)^{n-1}}\right\}
    \]
    and so
    \[
        (b+\epsilon)^n-b^n
        =\epsilon\sum_{k=0}^{n-1}(b+\epsilon)^kb^{n-1-k}
        <\epsilon\cdot n(b+\epsilon)^{n-1}
        <\epsilon\cdot n(b+1)^{n-1}
        <a-b^n.
    \]
    Hence \(b<b+\epsilon<a\),
    and by \cref{prop:archimedean}
    there exists \(r\in\Qmath\) such that
    \(b<r<b+\epsilon\) and \(r^n<(b+\epsilon)^n<a\).
    This contradicts the choice of \(b\) being an upper bound of \(S\),
    hence \(b^n\) cannot be smaller than \(a\).

    Similarly, suppose that \(b^n>a\).
    Then there exists \(\epsilon>0\) such that
    \[
        \epsilon<\frac{b^n-a}{nb^{n-1}}
    \]
    and so
    \[
        b^n-(b-\epsilon)^n
        =\epsilon\sum_{k=0}^{n-1}b^k(b-\epsilon)^{n-1-k}
        <\epsilon\cdot nb^{n-1}
        <b^n-a.
    \]
    Hence \(a<(b-\epsilon)^n<b^n\).
    But then for any \(r\in S\),
    \(r^n<a<(b-\epsilon)^n\) implies that
    \(b-\epsilon<b\) is also an upper bound of \(S\).
    This contradicts the choice of \(b\)
    being a \emph{least} upper bound of \(S\),
    hence \(b^n\) cannot be greater than \(a\).

    Since both cases where \(b^n<a\) and \(b^n>a\)
    result in a contradiction,
    we conclude that \(b^n=a\).
    That is,
    there exists a positive \(n\)-th root of \(a>0\).
    We also note that this root is unique;
    since \(0<b_1<b_2\) would result in \(b_1^n<b_2^n\).
    We denote this unique positive number as \(\denote{\sqrt[n]{a}}\).
\end{example}

Comparing this to the fact that \(\sqrt{2}\) is not a rational number,
we see that
the least upper bound property is much stronger
than the density condition.

It might also be worthwhile to define products of \(\Rmath\),
and how to measure distances in them.

\begin{definition}
    \label{def:Rn}
    For any positive integer \(\Nmath\),
    we call the Cartesian product
    \[
        \underbrace{\Rmath\times\dots\times\Rmath}_{n}
    \]
    as the \define{real coordinate space},%
    \footnote{
        However, we will seldom use the phrase
        ``real coordinate space,''
        mainly because
        the notation \(\Rmath^n\) is much more concise.
    }
    and denote it as \(\Rmath^n\).
    Note that the elements of \(\Rmath^n\)
    are \(n\)-tuples in the form of \(\trp*{a_1,\dots,a_n}\),
    where \(a_i\in\Rmath\) for each \(i=1,\dots,n\).%
    \footnote{
        The meaning of a small \textbf t at the top right
        will be explained later in linear algebra.
    }
    These are sometimes called \textdef{points} in \(\Rmath^n\).

    There are two operations for \(\Rmath^n\).
    Addition takes two elements
    \((x_1,\dots,x_n)\) and \((y_1,\dots,y_n)\) from \(\Rmath^n\)
    and returns the value
    \[
        (x_1,\dots,x_n)+(y_1,\dots,y_n)
        =(x_1+y_1,\dots,x_n+y_n).
    \]
    Scalar multiplication takes
    an element \((x_1,\dots,x_n)\) from \(\Rmath^n\)
    and an element \(a\) from \(\Rmath\)
    and returns the value
    \[
        a(x_1,\dots,x_n)
        =(ax_1,\dots,ax_n).
    \]
    In other words,
    addition and scalar multiplication are defined \emph{componentwise}.
\end{definition}

One note:
some people tend to distinguish notations
between numbers and \(n\)-tuples (or vectors in the later terminology).
While it is worthwhile for a variety of cases
such as linear algebra or numerical analysis,
I found it quite hard and meaningless
to practice different handwriting styles
for different fonts supported by \TeX.

\begin{definition}
    \label{def:abs}
    Given an element \(a=\trp*{a_1,\dots,a_n}\in\Rmath^n\),
    the \define{absolute value} of \(a\)
    is defined as
    \[
        \sqrt{\sum_{k=1}^{n}a_k^2},
    \]
    and is denoted as \(\denote{\abs a}[\abs{}]\).
\end{definition}

The absolute value plays a critical role in analysis
and is an important motivation for notions of magnitudes.
It has some important properties listed below.

\begin{observation}
    \label{obv:absnorm}
    For any \(\Rmath^n\),
    the map \(a\in\Rmath^n\mapsto\abs a\) has the following properties:
    \begin{nlist}
        \item For any \(x\in\Rmath^n\), \(\abs x\ge 0\);
        moreover, \(\abs x=0\) if and only if \(x=0\).
        \item For any \(x\in\Rmath^n\) and \(a\in\Rmath\),
        \(\abs{ax}=\abs a\abs x\).
        \item For any \(x,y\in\Rmath^n\),
        \(\abs{x+y}\le\abs{x}+\abs{y}\).
    \end{nlist}
\end{observation}
\begin{myproof}
    \begin{nlist}
        \item Let \(x=(x_1,\dots,x_n)\in\Rmath^n\).
        Then since \(x_i^2\ge 0\) for each \(i=1,\dots,n\),
        \(\sum_{i=1}^{n}x_i^2\ge 0\) and so is \(\abs x\).

        To show the latter part,
        suppose that \(\abs x=0\).
        Then since \(x_i^2\ge 0\)
        and \(x_i^2=0\) if and only if \(x_i=0\),
        we have that \(x_i=0\) for each \(i=1,\dots,n\)
        and thus \(x=0\).
        The converse is trivial.
        
        \item Let \(x=(x_1,\dots,x_n)\in\Rmath^n\) and \(a\in\Rmath\).
        Then
        \[
            \abs{ax}
            =\sqrt{\sum_{i=1}^{n}(ax_i)^2}
            =\sqrt{a^2\sum_{i=1}^{n}x_i^2}
            =\abs a\abs x.
        \]

        \item Let \(x=(x_1,\dots,x_n)\), \(y=(y_1,\dots,y_n)\)
        be in \(\Rmath^n\).
        Then
        \begin{align*}
            (\abs x+\abs y)^2-\abs{x+y}^2
            &=\abs x^2+2\abs x\abs y+\abs y^2-\abs{x+y}^2 \\
            &=2\abs x\abs y+\sum_{i=1}^{n}x_i^2+y_i^2-(x_i+y_i)^2
            =2\abs x\abs y-2\sum_{i=1}^{n}x_iy_i,
        \end{align*}
        and it suffices to prove that
        the last formula is nonnegative.
        \rightqed
    \end{nlist}
\end{myproof}

Since the inequality we want to prove at the last part
is of great importance,
we state it separately.
It even has a name on it --- the Cauchy-Schwarz inequality.

\begin{theorem}[Cauchy-Schwarz Inequality]
    \label{thm:CSI}

    Let \(x\) and \(y\) be points in \(\Rmath^n\)
    with \(x=(x_1,\dots,x_n)\) and \(y=(y_1,\dots,y_n)\).
    Then
    \[
        \sum_{i=1}^{n}x_iy_i\le\abs x\abs y.
    \]
\end{theorem}
\begin{myproof}
    For some \(t\in\Rmath\),
    consider the formula
    \[
        0
        \le\sum_{i=1}^{n}(x_i+ty_i)^2
        =\sum_{i=1}^{n}x_i^2
            +2t\sum_{i=1}^{n}x_iy_i+t^2\sum_{i=1}^{n}y_i^2.
    \]
    Since \(x\) and \(y\) are fixed points
    and the rightmost formula is nonnegative regardless of \(t\),
    the discriminant
    \[
        \dparen*{\sum_{i=1}^{n}x_iy_i}^2
            -\sum_{i=1}^{n}x_i^2\sum_{i=1}^{n}y_i^2\le 0,
    \]
    thus we are done.
\end{myproof}

There are numerous ways to prove this equality.
Right now, the proof relies heavily on the fact
that \(\Rmath\) is an ordered field.
However, Cauchy-Schwarz inequality and its variations
will appear constantly in this text;
after learning new ideas,
it is noteworthy to look at how the proof changes.