\setsection{Sequences in Metric Spaces}
\label{sec:mseq}

Normally, many textbooks move on to
notions of compactness and connectedness
after defining open sets.
However, it is also worthwhile to inspect sequences,
since they play a crucial role in metric spaces.

We first recall the definition of a sequence from
\cref{chap:numbers}.

% TODO: insert definition of sequence

Now since we have the notion of distance,
we can finally define what it means for a sequence to converge.

\begin{definition}
    \label{def:convseq}
    Let \(X\) be a metric space,
    and \(\seq{x_n}{n}\) be a sequence in \(X\).
    Then the sequence \(\seq{a_n}{n}\) is said to
    \define[sequence!convergent]{converge to \(x\in X\)}
    if there exists \(x\in X\) such that
    for any \(\epsilon>0\), there exists \(N>0\) such that
    \(d(x_n,x)<\epsilon\) for any \(n>N\).
    We denote as
    \[
        \denote{\lim_{n\to\infty}}x_n=x,
    \]
    or \(x_n\to x\) as \(n\to\infty\) in the middle of text.
    The latter is read as
    ``\(x_n\) converges to \(x\)
    as \(n\) tends to (or approaches) infinity,''
    while \(x\) is called
    a \define[sequence!limit]{limit} of \(\seq{a_n}{n}\).
    
    On the contrary,
    if \(\seq{x_n}{n}\) does not converge to any \(x\in X\),
    then it is said to
    \define[sequence!divergent]{diverge}.
\end{definition}

It is also common to call a sequence \(\seq{a_n}{n}\)
\textdef{convergent} or \textdef{divergent}.

According to \textcite{Bre07},
it was Archimedes who introduced the notion above
to describe convergence.
This was used when he was describing the method of exhaustion,
and needed to argue that
the sum of the area of triangles approaches the area of a parabola.
The same idea will arise later when we discuss integration.

Note that the definition is written ``backwards'' in some sense.
Normally, one would imagine that to determine a limit of a sequence,
one would inspect the value of its terms while increasing the index.
However, the formal definition of a limit is to
first fix the candidate,
and then showing that every term is close enough to that candidate
for sufficiently large indices.
For example, consider the sequence \(\seq{a_n=1/n}{n}\).
The former approach first watches the values \(1,1/2,1/3,\dots\)
and states that it gets arbitrarily close to zero.
The latter approach first makes zero as a candidate of a limit,
and then shows that the terms are close to zero except for finite terms.
However, it is noteworthy that the former approach is still relevant,
since we need to determine the candidate for our limit.

Now a very natural question arises:
if a sequence converges to a limit, is it unique?

\begin{observation}
    \label{obv:uniqlim}
    Let \(X\) be a metric space
    and \(\seq{x_n}{n}\) be a sequence in \(X\).
    If \(x\) and \(x'\) are limits of the sequence \(\seq{x_n}{n}\),
    then \(x=x'\).
\end{observation}
\begin{myproof}
    Let \(x\) and \(x'\) be limits of \(\seq{x_n}{n}\),
    and fix \(\epsilon>0\).
    Then there exist \(N,N'>0\) such that
    \(d(x_n,x)<\epsilon/2\) for all \(n>N\)
    and \(d(x_n,x')<\epsilon/2\) for all \(n>N'\).
    Therefore
    \[
        d(x,x')
        \le d(x,x_n)+d(x',x_n)
        <\frac{\epsilon}{2}+\frac{\epsilon}{2}
        =\epsilon
    \]
    for any \(n>\max\{N,N'\}\).
    This must hold for any \(\epsilon>0\);
    thus \(d(x,x')=0\) and so \(x=x'\).
\end{myproof}

Of course, there are other important properties of convergent sequences.

\begin{observation}
    \label{obv:convseq}
    Let \(X\) be a metric space and \(x\in X\).
    \begin{nlist}
        \item The sequence \(\seq*{x_n}{n}\) converges to \(x\)
        if and only if every open ball centered at \(x\)
        contains all but finitely many points of \(\seq*{x_n}{n}\).

        \item If a sequence \(\seq*{x_n}{n}\) converges,
        then it is bounded.
        That is,
        there exists an \(R>0\) such that
        \(d(x_m,x_n)<R\) for any \(m,n\).

        \item If a sequence \(\seq*{x_n}{n}\) converges to \(x\),
        then \(x\) is a limit point
        of the set \(\set*{x_n}{n\in \Nmath}\).
    \end{nlist}
\end{observation}
\begin{myproof}
    \begin{nlist}
        \item Suppose that \(\seq*{x_n}{n}\) converges to \(x\).
        Then given any open ball \(D(x,R)\) for \(R>0\),
        there exists \(N>0\) such that
        \(n>N\) implies \(d(x,x_n)<R\); that is, \(x_n\in D(x,R)\).

        Conversely, suppose that every open ball centered at \(x\)
        contains all but finitely many points of \(\seq*{x_n}{n}\).
        Then given any \(\epsilon>0\),
        since the open ball \(D(x,\epsilon)\) must contain
        all but finitely many terms of the sequence \(\seq*{x_n}{n}\);
        thus there must exist \(N>0\) such that
        \(x_n\in D(x,\epsilon)\) holds for all \(n>N\).

        \item Suppose that
        the sequence \(\seq*{x_n}{n}\) converges to \(x\).
        Then there exists \(N>0\) such that
        \(d(x_n,x)<1\) for all \(n>N\);
        thus by taking
        \[
            R
            =\max\{
                d(x_1,x_2),d(x_1,x_3),\dots,d(x_{N-1},x_N),
                d(x_1,x),d(x_2,x),\dots,d(x_N,x)
            \}+1
        \]
        we have that \(d(x_m,x_n)<R\) for any \(m,n\).

        \item Suppose that
        the sequence \(\seq*{x_n}{n}\) converges to \(x\),
        and let \(S=\set*{x_n}{n\in\Nmath}\).
        Then given any \(\epsilon>0\),
        the convergence of \(\seq*{x_n}{n}\) implies that
        the open ball \(D(x,\epsilon)\) contains all but finite \(x_n\);
        thus there must exist a point in \(S\) distinct from \(x\).
        Therefore \(x\) is indeed a limit point of \(S\).
        \rightqed
    \end{nlist}
\end{myproof}

Item (3) of the above observation shows the similarity
between the terms `limits' and `limit points.'
We have seen that
the limit of a sequence is a limit point of the image of that sequence.
Sadly, by observing the sequence \(\seq*{a_n=(-1)^n}{n}\) in \(\Rmath\)
we notice that the converse is not necessarily true.
Even though \(1\) and \(-1\) are limit points
of the image of \(\seq*{a_n}{n}\),
the sequence \(\seq*{a_n}{n}\) clearly diverges.

However, we still can find some terms that converge to them.
Can we generalize this to metric spaces?
To do so we first need a notion of picking some terms of a sequence.

\begin{definition}
    \label{def:subseq}
    Let \(n_1,n_2,n_3,\dots\) be natural numbers
    of (strictly) increasing order.
    Then given a sequence \(\seq*{x_n}{n}\),
    the new sequence \(\seq*{x_{n_k}}{k}\) is called
    a \define[sequence!sub-]{subsequence} of \(\seq*{x_n}{n}\).
\end{definition}

\begin{proposition}
    \label{prop:limptsubseq}
    Let \(\seq*{x_n}{n}\) be a sequence in a metric space \(X\),
    and \(x\) be a limit point of the set \(\set*{x_n}{n\in\Nmath}\).
    Then there exists a subsequence of \(\seq*{x_n}{n}\)
    which converges to \(x\).
\end{proposition}
\begin{myproof}
    Since \(x\) is a limit point of the set \(\set*{x_n}{n\in\Nmath}\)
    (which we denote by \(E\)),
    given any \(k\in\Nmath\)
    there must exist some point \(x_{n_k}\in E\) with
    \(n_k>n_{k-1}\) and \(d(x,x_{n_k})<1/k\).
    We claim that the subsequence \(\seq*{x_{n_k}}{k}\)
    converges to \(x\).
    
    Now given any \(\epsilon>0\),
    there exists \(K\in\Nmath\) such that \(d(x,x_{n_K})<1/K<\epsilon\);
    thus for any \(k>K\) we have
    \[
        d(x_{n_k},x)
        <\frac{1}{k}
        <\frac{1}{K}
        <\epsilon,
    \]
    and we conclude that
    \(\seq*{x_{n_k}}{k}\) indeed converges to \(x\).
\end{myproof}

% If we depict a converging sequence in our minds,
% we note that the terms of the sequence gets closer to each other.
% This family of sequences play an important role
% in the study of analysis,
% and thus is named after Cauchy.
% We have already seen this sequence in \cref{chap:numbers}.

% % TODO : restate definition of cauchy sequence

% Note that
% when working with Cauchy sequences,
% we only need to inspect the \emph{terms} of the sequence,
% whereas when working with convergent sequences,
% we need to inspect the limit, which is an extra value.
% Recalling the definition of a Cauchy sequence,
% we restate the above discussion as follows.

% \begin{observation}
%     \label{obv:convcauchy}
%     Let \(\seq{x_n}{n}\) be a convergent sequence
%     in a metric space \(X\).
%     Then \(\seq{x_n}{n}\) is a Cauchy sequence.
% \end{observation}
% \begin{sketch}
%     As above,
%     the sequence \(\seq{x_n}{n}\) gets closer to its limit.
%     Hence the distance of the terms tends to zero.
% \end{sketch}
% \begin{myproof}
%     Let \(x\) be the limit of \(\seq{x_n}{n}\),
%     and fix \(\epsilon>0\).
%     Since \(\seq{x_n}{n}\) converges to \(x\),
%     there exists \(N\) such that
%     \(n>N\) implies \(d(x_n,x)<\epsilon/2\).
%     Therefore for any \(m>n>N\),
%     \[
%         d(x_n,x_m)
%         \le d(x_n,x)+d(x,x_m)
%         <\frac{\epsilon}{2}+\frac{\epsilon}{2}
%         =\epsilon
%     \]
%     and thus \(\seq{x_n}{n}\) is indeed a Cauchy sequence.
% \end{myproof}

% Now we hope the converse to be true.
% That is, we will observe
% whether every Cauchy sequence converges to some point.
% Sadly, this is not true.
% Take, for example,
% the sequence
% \[
%     1, \qquad
%     1.4, \qquad
%     1.41, \qquad
%     1.414, \qquad
%     1.4142, \qquad
%     1.41428, \qquad
%     \cdots
% \]
% formed by the decimal expression of \(\sqrt 2\).
% This is a sequence in \(\Qmath\),
% but the result \(\sqrt 2\) is not in \(\Qmath\).

% Since we know that the converse of \cref{obv:convcauchy} is not true,
% the next question would be:
% What conditions do we need to make every Cauchy sequence converge?
% We first name the condition that we want.

% \begin{definition}
%     \label{def:complete}
%     We call a metric space \(X\)
%     \define[metric space!complete]{complete}
%     if every Cauchy sequence in \(X\) converges.
% \end{definition}