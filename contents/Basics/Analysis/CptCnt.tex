\setsection{Compactness}
\label{sec:mcptcnt}

Now we define a notion for ``compactness.''
The history of the notion for compactness is not that easy
and the reasoning behind the terminology is quite vague,
as stated by \textcite{cpt-hist}.
However, it is said that
there are mainly three notions of compactness.

The first definition of compactness
comes from our early definitions of a limit point.

\begin{definition}
    \label{def:limptcpt}
    Let \(X\) be a metric space.
    If any infinite subset \(E\) of \(X\) has a limit point in \(X\),
    then \(X\) is said
    to be \define[compactness!limit-point]{limit-point compact},
    or to have the \define{Bolzano-Weierstrass property}.
\end{definition}

Thus one cannot call \(\Rmath\) limit-point compact,
since the infinite subset \(\Nmath\) does not have a limit point.
Also, one cannot call the open interval \((a,b)\) limit-point compact,
since the set
\[
    E
    =\set[\bigg]{a+\frac{b-a}{n}}{n\in\Nmath}
\]
does not have a limit point in \((a,b)\).

\begin{definition}
    \label{def:seqcpt}
    Let \(X\) be a metric space.
    If any sequence in \(X\) has a convergent subsequence,
    then \(X\) is said
    to be \define[compactness!sequential]{sequentially compact}.
\end{definition}

\begin{definition}
    \label{def:mcpt}
    Let \(X\) be a metric space.
\end{definition}