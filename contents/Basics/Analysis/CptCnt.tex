\setsection{Compactness}
\label{sec:mcptcnt}

Now we define a notion for ``compactness.''
The history of the notion for compactness is not that easy
and the reasoning behind the terminology is quite vague,
as stated by \textcite{cpt-hist}.
However, it is said that
there are mainly three notions of compactness.

The first definition of compactness
comes from our early definitions of a limit point.

\begin{definition}
    \label{def:limptcpt}
    Let \(X\) be a metric space.
    If any infinite subset \(E\) of \(X\) has a limit point in \(X\),
    then \(X\) is said
    to be \define[compactness!limit-point]{limit-point compact},
    or to have the \define{Bolzano-Weierstrass property}.
\end{definition}

Thus one cannot call \(\Rmath\) limit-point compact,
since the infinite subset \(\Nmath\) does not have a limit point.
Also, one cannot call the open interval \((a,b)\) limit-point compact,
since the set
\[
    E
    =\set[\bigg]{a+\frac{b-a}{n}}{n\in\Nmath}
\]
does not have a limit point in \((a,b)\).
This may be called compact in a sense,
since \(\Rmath\) and \(E\) cannot be compact.

But by the aid of \cref{prop:limptsubseq},
one can consider another definition of compactness,
as stated below.
Note that
this is not exactly the same as limit point compactness,
since the infinite set in \cref{def:limptcpt} might be uncountable,
and thus cannot be represented as the image of a sequence.

\begin{definition}
    \label{def:seqcpt}
    Let \(X\) be a metric space.
    If any sequence in \(X\) has a convergent subsequence,
    then \(X\) is said
    to be \define[compactness!sequential]{sequentially compact}.
\end{definition}

Nevertheless,
sequential compactness also plays an important role in topology.

The final definition we state for compactness is a bit different.
Instead of dealing with subsets and points in a space,
we consider yet another property of a set.

\begin{definition}
    \label{def:opencover}
    Let \(X\) be a metric space,
    and \(E\subset X\).
    If \(\mathcal G=\indexset*{G_i}{i\in I}{}\) is a family of
    open sets in \(X\) such that
    \(E\subset\bigcup\mathcal G\),
    then \(\mathcal G\) is called
    an \define{open cover} of \(E\).
    If \(\mathcal H\subset\mathcal G\) also satisfies
    \(E\subset\bigcup\mathcal H\),
    then \(\mathcal H\) is called
    a \define{subcover} of \(\mathcal G\).
\end{definition}

Now we define our final definition of compactness.

\begin{definition}
    \label{def:mcpt}
    Let \(X\) be a metric space
    and \(E\subset X\).
    If any open cover of \(E\) has a finite subcover,
    then \(E\) is said to be
    \define[compactness!open-cover]{open-cover compact}.
\end{definition}

Since the modern definition of compactness comes from \cref{def:mcpt},
we shall call open-cover compactness as ``compact''
when there is no ambiguity.